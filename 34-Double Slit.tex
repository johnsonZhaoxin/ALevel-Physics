\documentclass[a4paper]{tufte-handout}
% for debugging purposes -- displays the margins
%\geometry{showframe}
%\usepackage{float} %不知道和什么冲突了。
\usepackage{amsmath,amsfonts,amssymb,mathtools}
\newcommand{\icol}[1]{% inline column vector
  \left(\begin{smallmatrix}#1\end{smallmatrix}\right)%
}


\newcommand{\irow}[2]{ % 行向量,输出 xi+yj+zk的形式,存在的问题是不能自动根据负数调整为-号
  #1\mathbf{i}+#2\mathbf{j}%+#3\mathbf{k}
}
% \usepackage{geometry} %这个包已经在handout.cls使用过了,如果再调用一次会出问题
% \geometry{left=2cm,right=2cm,top=3cm,bottom=3cm}

%\usepackage[LGR]{fontenc} %使用希腊字符。

\usepackage{siunitx}
\DeclareSIUnit\torr{torr} %声明新的单位torr
\usepackage{mhchem}

\usepackage{BOONDOX-cal} %为了花体的emf符号
\usepackage[normalem]{ulem} %下划线
\usepackage{wrapfig}
\usepackage{caption}
\usepackage{float}

%定理的运用
\usepackage[english]{babel}
\newtheorem{theorem}{Theorem}[section]      %定理
\newtheorem{corollary}{Corollary}[theorem]  %结论
\newtheorem{lemma}[theorem]{Lemma}          %引理
\newtheorem{definition}[theorem]{Definition}%定义
% ------------------------------------------------------------------------------
% load hyperref to use hyperlinks
% ------------------------------------------------------------------------------
\usepackage{xcolor}
\definecolor{r1}{HTML}{FF8674}
\definecolor{b1}{HTML}{17ABDD}
\definecolor{p1}{HTML}{D4B6D6}
\definecolor{g1}{HTML}{70E2CB}
\definecolor{o1}{HTML}{DFA743}


\usepackage{hyperref}
\hypersetup{
      colorlinks=true,
      linkcolor=black,
      filecolor=cyan,
      urlcolor=b1,
      citecolor=g1,
}
% Set up the images/graphics package
\usepackage{graphicx}
%\graphicspath{{./auxi/}}
\setkeys{Gin}{width=0.7\linewidth,totalheight=0.3\textheight,keepaspectratio}

% The following package makes prettier tables.  We're all about the bling!
\usepackage{booktabs}

% The units package provides nice, non-stacked fractions and better spacing
% for units.
\usepackage{units}
\usepackage{nicefrac} %比较好看的斜体分号
\usepackage{siunitx}
\sisetup{separate-uncertainty}%产生的效果是是20+3cm 这样
% The fancyvrb package lets us customize the formatting of verbatim
% environments.  We use a slightly smaller font.
\usepackage{fancyvrb}
\fvset{fontsize=\normalsize}

% Small sections of multiple columns
\usepackage{multicol}

% Better variable font
\usepackage{mathptmx}

% todolist
\usepackage{enumitem}
\newlist{todolist}{itemize}{2}
\setlist[todolist]{label=$\square$} %因为美元符号的问题。需要手动添加美元\square美元

\usepackage{tikz}
\usepackage{circuitikz}

%beautifulbox
\usepackage{tcolorbox} %带背景色的盒子用于放置Summary,Task,还有Practice
\tcbuselibrary{breakable}
\tcbset{width=\textwidth} %默认盒子的宽度
\newenvironment{TaskBox} %任务盒子
{\begin{tcolorbox}[breakable,colback=b1!30,colframe=b1,title=Task]} {\end{tcolorbox}}
\newenvironment{ExampleBox} %Practice盒子
{\begin{tcolorbox}[breakable,colback=g1!30,colframe=g1,title=Example]} {\end{tcolorbox}}
\newenvironment{SummBox}
{\begin{tcolorbox}[breakable,colback=r1!30,colframe=r1,title=Summary]} {\end{tcolorbox}}


% These commands are used to pretty-print LaTeX commands
%\newcommand{\doccmd}[1]{\texttt{\textbackslash#1}}% command name -- adds backslash automatically
%\newcommand{\docopt}[1]{\ensuremath{\langle}\textrm{\textit{#1}}\ensuremath{\rangle}}% optional command argument
%\newcommand{\docarg}[1]{\textrm{\textit{#1}}}% (required) command argument
%\newenvironment{docspec}{\begin{quote}\noindent}{\end{quote}}% command specification environment
%\newcommand{\docenv}[1]{\textsf{#1}}% environment name
%\newcommand{\docpkg}[1]{\texttt{#1}}% package name
%\newcommand{\doccls}[1]{\texttt{#1}}% document class name
%\newcommand{\docclsopt}[1]{\texttt{#1}}% document class option name

\def\d{{\mathrm{d}}}

% 强制所有段落不缩进
\setlength{\parindent}{0pt}%没有起作用,日
\title{}
\author{Sanjin Zhao}
\date{20th Nov, 2022}  % if the \date{} command is left out, the current date will be used


\begin{document}
\maketitle% this prints the handout title, author, and date
%\printclassoptions
\section*{Learning Outcome}
I highly recommend you to finish this checklist to determine whether you've achieved the learning objectives.
\begin{todolist}
  \item explain the meaning of interference, path difference and coherence
  \item explain double slit experiment
  \item explain the rationale of diffraction gratings
  \item understand the conditions required if two-source interference fringes are to be observed
  \item recall and use $\lambda = \frac{ax}{D}$ for double-slit interference using light
  \item recall and use $d\sin\theta =n \lambda$ for a diffracting grating
  \item use a diffraction grating to determine the wavelength of light.
\end{todolist}
\clearpage

\section{Leadin}
Now let's meet Thomas Young again, but this time for his magical double slit experiment.
\begin{marginfigure}
\includegraphics[width=5cm]{auxi/young.jpg}
\caption{Thomas Young\\1773-1829}
\end{marginfigure}
Thomas is a firm support of the theory that light is actually a wave, in order to prove his paper, double slit experiemnt is his most famous, but some might find it perplexing, and strongest weapon.

\section{review of wave interference}
Wave ripples will interfere with each other at every position that the two waves pass. At certain place, the waves at such places are \textbf{in/out}phase, and  will always cancel out or in other words, \textbf{constructive/destructive} interference occurs at such position, the amplitude at such position is \textbf{maximized/minimized}.
\begin{figure}[h]
\centering
\includegraphics[width=0.8\linewidth]{auxi/ripple.jpg}
\caption{inteference occurs with ripples}
\label{fig:ripple}
\end{figure}

\begin{TaskBox}
Define the following concept:\\
\begin{itemize}
  \item principle of superposition
  \item in/out of phase or antiphase
  \item coherent source/waves
\end{itemize}
\tcblower
Label some position in Fig.\ref{fig:ripple} where constructive/destructive interference occurs 
\end{TaskBox}

\section{Double Slit Experiment}
Young's veiwpoint is direct, if light is wave, then interference will occur, and since the amplitude of light is related to the \uline{\hspace{1in}} of the light, thus bright and dark patterns will occur at some positions. And the nature proves that it is exactly correct!
\begin{marginfigure}
\includegraphics[width=5cm]{auxi/doubleslit1.jpg}
\caption{Tough ray model is not good enough to explain, but the maxima is labelled}
\label{fig:ray-doubleslit}
\end{marginfigure}

\subsection{coherence}
As talked before, the two waves must be \emph{coherent} in order to produce a interference pattern, this premise also holds true for the light. If they were not coherent sources, the interference pattern would be constantly changing, far too fast for our eyes to detect.
\begin{figure}[h]
\centering
\includegraphics[width=0.8\linewidth]{auxi/coherent.jpg}
\caption{the purpose of using double slit is to make sure that the waves are coherent}
\label{fig:coherent}
\end{figure}

\subsection{diffraction after slits}
When light passes the slit or obstacles with width roughly equals to the wavelength of the ligft, diffraction will occur, so one thing to mention is that diffraction is included in the double slit experiment, that's so common that most people will ignores.
\begin{figure}[h]
\centering
\includegraphics[width=0.8\linewidth]{auxi/diffraction.jpg}
\caption{The experiment could also be used to demonstrate diffraction, as long as one slit is sealed}
\label{fig:diffraction in double slit}
\end{figure}

\begin{TaskBox}
Identify what parts in Fig.\ref{fig:diffraction in double slit} shows diffraction
\end{TaskBox}

\subsection{The result}
Since we use screens after the slits to capture the result, what we observe is quite similar to diffraction. Bright and Dark fringes will occur, but in a much more uniformly distribution. The intensity graph is what we need for a quantitatively
\begin{figure}[h]
\centering
\includegraphics[width=0.8\linewidth]{auxi/Double-Slit-Experiment.png}
\caption{Ray model is much more frequently used to explain the double slit experiment}
\end{figure}

\begin{SummBox}
The result of double slit experiment is quite phenomenal, several important conclusion should be memorized, it can not be greater if you've see it by your own eyes.

\begin{marginfigure}
\includegraphics[width=4cm]{auxi/doubleslit2.jpg}
\caption{one imperfect thing is that the border of bright and dark pattern is not clear, but this would be enough to prove that light is a wave}
\end{marginfigure}

\begin{itemize}
  \item bright and dark pattern will show on the screen after the slit
  \item but measuring the intensity, it will be more accurate to see the intensity will vary with the position, some at peak, some at trough.
  \item this does not mean interference occurs only at the screen
\end{itemize}
\end{SummBox}


\subsection{Path Difference}
The most powerful evidence of wave nature of light in double slit experiment is the central brightest lines of light, which lies exactly behind the central of the two slits. Such exsistance fully refute the particle theory of light\footnote{why?}, and appears exactly the same as water ripple. 

One of the most important task is to explain how double slit works. Now let's introduce the concept of \emph{path difference}
\begin{figure}[h]
\centering
\includegraphics[width=0.8\linewidth]{auxi/pathdifference.jpg}
\caption{light always propagate in a straight line}
\label{fig:path difference-ray}
\end{figure}

At point $A$, it is at the center, and it is a bright fringe. Because, two rays of light arrive at $A$, one from slit 1 and the other from slit 2. Point $A$ is \textbf{equidistant} from the two slits, The \emph{path difference} between the two rays of light is \textbf{zero/0}. thus, they will be \textbf{in phase} when they arrive at $A$.

\begin{TaskBox}
Explain why the fringe at $C$ is still bright, State the path difference in terms of the wavelength.
\tcblower
Explain why the fringe at $B$ is dark, State the path difference in terms of the wavelength.
\end{TaskBox}

\section{Formula in Double Slit Experiment}
\subsection{quailitative conclusion}
Use \href{https://phet.colorado.edu/en/simulations/wave-interference}{phet simulation}, adjust the wavelength of the light $\lambda$, slit separation $a$, and the slit-to-screen distance $D$ to change the pattern of the fringes separation $x$.

\begin{SummBox}
\begin{itemize}
  \item As $\lambda$ increase, the fringe separation will become \uline{larger/smaller}.
  \item As $a$ increase, the fringe separation will become \uline{larger/smaller}.
  \item As $D$ increase, the fringe separation will become \uline{larger/smaller}.
\end{itemize}
\end{SummBox}

\subsection{Equation}
The famous double slit equation is summarized below:
\[
  \lambda = \frac{ax}{D}
\]
This is the important quantitative formula which relates the wavelength and the fringe separation. The invisible wavelength can be expressed and determined by the measuarable length.

\section{Diffraction Gratings}
Despite the name, the more important part in which diffraction gratings plays is the interference of light rather than the diffraction. There are two types of gratings.

\subsection{Types}
Transmission diffraction gratings is similar to the double slit slide used in Young's experiment, but there are far more slits which are evenly distributed among the slide. Usually it is described by ``2000 lines per centimetre''
\begin{marginfigure}
\centering
\includegraphics[width=5cm]{auxi/grating1.jpg}
\caption{A transmission diffraction grating}
\end{marginfigure}
\begin{TaskBox}
Determine the spacing distance, in \si{\micro\meter}, of the grating described by ``2000 lines per centimetre''
\end{TaskBox}

Another type is reflection diffraction gratings, in which regular and uniform line patterns are cut on a reflective material, the most famous is the CD or DVD
\begin{marginfigure}[3cm]
\centering
\includegraphics[width=5cm]{auxi/grating2.jpg}
\caption{A reflection diffraction grating}
\end{marginfigure}

\begin{figure}[h]
\centering
\includegraphics[width=0.9\linewidth]{auxi/cd-dvd.png}
\caption{CD and DVD are naturally a good relfection diffraction grating}
\end{figure}

\begin{marginfigure}
\centering
\includegraphics[width=5cm]{auxi/cdcolor.jpg}
\caption{This is the 0th order  maximum on CD}
\end{marginfigure}

\begin{TaskBox}
Compare the two types of diffraction gratings, state the name, phenomena occured.
\tcblower
\vspace{1in} 
\end{TaskBox}

\subsection{Explanation}
Since there are lots of slits in a diffraction gratings, when light pass through every single slits, it will \uline{\hspace{1in}} and spread all around, cause \uline{\hspace{1in}} of light. 
\begin{figure}[h]
\centering
\includegraphics[width=0.8\linewidth]{auxi/diffractionresult2.jpg}
\caption{The simulated result of diffraction gratings}
\end{figure}

\begin{marginfigure}[-3cm]
\includegraphics[width=3.5cm]{auxi/diffractionresult4.png}
\caption{The real result of diffraction gratings}
\label{fig:realresult}
\end{marginfigure}

But the more intriguing fact is that the lights after each slit are parallel. As shown in Fig.\ref{fig:explanation}.
\begin{figure}[h]
\centering
\includegraphics[width=0.8\linewidth]{auxi/explanation.jpg}
\caption{Parallel light after the gratings interfere with each other}
\label{fig:explanation}
\end{figure}

For the central maxima in which the angle $\theta=0$, or the maxima label is $n=0$, all the rays of light travel the same distance. Hence, it will arrive in phase. That's the reason for the 0th-order maximum.

\begin{TaskBox}
Using Fig.\ref{fig:explanation}b. Explain the formation of the 1st maximum.
\tcblower
Explain why the 0th maxima is the white instead of iridescent light in Fig.\ref{fig:realresult}.
\end{TaskBox}

\subsection{Formula}
Diffraction grating is analyzed using interference formulas, not diffraction formulas. And let's deduce the formula of interference in diffraction gratings.
\begin{marginfigure}
\centering
\includegraphics[width=5cm]{auxi/dsintheta.png}
\caption{path difference are still useful, but in a parallel perspective}
\label{fig:pathd in gratings}
\end{marginfigure}
The angle \uline{$\theta$} at which the maximum occurs and its \uline{\emph{order},$n$} is related. For example, the greater the order is, the \uline{greater/smaller} $\theta$ will be. And according to relationship between path difference. It will be easily deduced that:
\[
  d\sin\theta=n\lambda
\]

\begin{TaskBox}
State the meaning of each physical quantities.
\tcblower
Rearrange the formula to obtain an equation for $\lambda$
\end{TaskBox}
\end{document}

