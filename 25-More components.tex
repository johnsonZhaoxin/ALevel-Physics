\documentclass[a4paper]{tufte-handout}
% for debugging purposes -- displays the margins
%\geometry{showframe}
%\usepackage{float} %不知道和什么冲突了。
\usepackage{amsmath,amsfonts,amssymb,mathtools}
\newcommand{\icol}[1]{% inline column vector
  \left(\begin{smallmatrix}#1\end{smallmatrix}\right)%
}


\newcommand{\irow}[2]{ % 行向量,输出 xi+yj+zk的形式,存在的问题是不能自动根据负数调整为-号
  #1\mathbf{i}+#2\mathbf{j}%+#3\mathbf{k}
}
% \usepackage{geometry} %这个包已经在handout.cls使用过了,如果再调用一次会出问题
% \geometry{left=2cm,right=2cm,top=3cm,bottom=3cm}

%\usepackage[LGR]{fontenc} %使用希腊字符。

\usepackage{siunitx}
\DeclareSIUnit\torr{torr} %声明新的单位torr


\usepackage{BOONDOX-cal} %为了花体的emf符号
\usepackage[normalem]{ulem} %下划线
\usepackage{wrapfig}
\usepackage{caption}
\usepackage{float}
\usepackage[version=4]{mhchem}

%定理的运用
\usepackage[english]{babel}
\newtheorem{theorem}{Theorem}[section]      %定理
\newtheorem{corollary}{Corollary}[theorem]  %结论
\newtheorem{lemma}[theorem]{Lemma}          %引理
\newtheorem{definition}[theorem]{Definition}%定义
% ------------------------------------------------------------------------------
% load hyperref to use hyperlinks
% ------------------------------------------------------------------------------
\usepackage{xcolor}
\definecolor{r1}{HTML}{FF8674}
\definecolor{b1}{HTML}{17ABDD}
\definecolor{p1}{HTML}{D4B6D6}
\definecolor{g1}{HTML}{70E2CB}
\definecolor{o1}{HTML}{DFA743}


\usepackage{hyperref}
\hypersetup{
      colorlinks=true,
      linkcolor=black,
      filecolor=cyan,
      urlcolor=b1,
      citecolor=g1,
}
% Set up the images/graphics package
\usepackage{graphicx}
%\graphicspath{{./auxi/}}
\setkeys{Gin}{width=0.7\linewidth,totalheight=0.3\textheight,keepaspectratio}

% The following package makes prettier tables.  We're all about the bling!
\usepackage{booktabs}

% The units package provides nice, non-stacked fractions and better spacing
% for units.
\usepackage{units}
\usepackage{nicefrac} %比较好看的斜体分号
\usepackage{siunitx}
\sisetup{separate-uncertainty}%产生的效果是是20+3cm 这样
% The fancyvrb package lets us customize the formatting of verbatim
% environments.  We use a slightly smaller font.
\usepackage{fancyvrb}
\fvset{fontsize=\normalsize}

% Small sections of multiple columns
\usepackage{multicol}

% Better variable font
\usepackage{mathptmx}

% todolist
\usepackage{enumitem}
\newlist{todolist}{itemize}{2}
\setlist[todolist]{label=$\square$} %因为美元符号的问题。需要手动添加美元\square美元

\usepackage{tikz}

%beautifulbox
\usepackage{tcolorbox} %带背景色的盒子用于放置Summary,Task,还有Practice
\tcbuselibrary{breakable}
\tcbset{width=\textwidth} %默认盒子的宽度
\newenvironment{TaskBox} %任务盒子
{\begin{tcolorbox}[breakable,colback=b1!30,colframe=b1,title=Task]} {\end{tcolorbox}}
\newenvironment{ExampleBox} %Practice盒子
{\begin{tcolorbox}[breakable,colback=g1!30,colframe=g1,title=Example]} {\end{tcolorbox}}
\newenvironment{SummBox}
{\begin{tcolorbox}[breakable,colback=r1!30,colframe=r1,title=Summary]} {\end{tcolorbox}}


% These commands are used to pretty-print LaTeX commands
%\newcommand{\doccmd}[1]{\texttt{\textbackslash#1}}% command name -- adds backslash automatically
%\newcommand{\docopt}[1]{\ensuremath{\langle}\textrm{\textit{#1}}\ensuremath{\rangle}}% optional command argument
%\newcommand{\docarg}[1]{\textrm{\textit{#1}}}% (required) command argument
%\newenvironment{docspec}{\begin{quote}\noindent}{\end{quote}}% command specification environment
%\newcommand{\docenv}[1]{\textsf{#1}}% environment name
%\newcommand{\docpkg}[1]{\texttt{#1}}% package name
%\newcommand{\doccls}[1]{\texttt{#1}}% document class name
%\newcommand{\docclsopt}[1]{\texttt{#1}}% document class option name

\def\d{{\mathrm{d}}}

% 强制所有段落不缩进
\setlength{\parindent}{0pt}%没有起作用,日
\title{Miscellaneous in Electricity}
\author{Sanjin Zhao}
\date{2nd Oct, 2022}  % if the \date{} command is left out, the current date will be used


\begin{document}
\maketitle% this prints the handout title, author, and date
%\printclassoptions
\section*{Learning Outcome}
I highly recommend you to finish this checklist to determine whether you've achieved the learning objectives.
\begin{todolist}
  \item understand that metals have \emph{delocalised electrons} to be conductive
  \item explain why temeperature could change the resistance, and the phenomenon of \emph{superconducitiviy}
  \item sketch and explain the $I$–$V$ characteristics for \emph{filament lamp}, \emph{diodes}, \emph{thermistor}
  \item sketch the temperature characteristic for an NTC thermistor
  \item sketch the light intensity characteristics for an light-dependent resistor(LDR)
\end{todolist}
\clearpage

\section{Leadin}
Resistor is just the start in electronics, there are more components to be invented in a row, let's dig more. 
\begin{marginfigure}
\includegraphics[width=5cm]{auxi/Componentes.JPG}
\caption{What are the names of the components above?}
\end{marginfigure}

\section{Essence of Conductivity}
In previous ``Currents'' chapter, the metals are conductive because a sea of electrons exsist in the cavity. and now such free electrons are called \emph{delocalised electron}

\subsection{Delocalised Electron}
The definition is as following
\begin{SummBox}
Delocalised Electron is
\vspace{0.5in}
\end{SummBox}

\subsection{Effect of Temperature on Resistance}
The relationship between resistance in a metal and temperature is that: usually, if the temperature becomes \uline{\hspace{1in}}, the resistance will also become \uline{\hspace{1in}}.

\begin{figure}[h]
\centering
\includegraphics[width=0.8\linewidth]{auxi/temperature.png}
\caption{a model used to explain the resistance}
\label{fig:model}
\end{figure}

From Fig.\ref{fig:model}


\subsection{Superconductivity}
If the temperature is low enough, is there any possibility that the resistance suddenly disappear? The answer is quite obvious, it do happen, such phenomenon is called \emph{superconductivity}\footnote{exp: }.

%卡末林·昂内斯最早发现了超导现象
Superconductivity was firstly discovered by \href{https://en.wikipedia.org/wiki/Heike_Kamerlingh_Onnes}{Heike Kamerlingh Onnes}, The critical temperature for mercury to be superconductive is \SI{4.1}{\K}. And usually the critical temperature, $T_c$, for metals is around \SIrange{0}{10}{\K}. So one of the cutting-edge field of research in superconductivity is high-temprature superconductivity or even room-temperature superconductivity.In \href{https://caoyuan.scripts.mit.edu}{Yuan CAO}'s publication ``\href{https://www.nature.com/articles/nature26160}{Unconventional superconductivity in magic-angle graphene superlattices}'', unconventional superconductivity might occur at the temperature of \SI{1.7}{\K}.
\begin{figure}[h]
\centering
\includegraphics[width=0.7\linewidth]{auxi/graphene_sheet.jpg}
\caption{an angle of \ang{1.1} is the magical angle}
\end{figure}

\begin{marginfigure}[+3cm] %偏移量
\centering
\includegraphics[width=4cm]{auxi/caoyuan.jpg}
\caption{Yuan CAO,Forbes 30 under 30 Asia}
\end{marginfigure}

\begin{TaskBox}
List the field that superconductivity could be applied.
\vspace{1in}
\end{TaskBox}

\subsection{Effect of Impurity on Resistance}
Alloys are the mixture of two or more metals, and ususally because the difference between the nuclei of different metal, the resistance might be raised as well. 
\begin{figure}[h]
\centering
\includegraphics[width=0.8\linewidth]{auxi/impurity.png}
\caption{impurity might also influence the resistance}
\end{figure}

\section{Filament Lamp}
A conductor that does not obey Ohm’s law is described as non-ohmic. An example is a filament lamp. Let's dig out the $I$-$V$ characteristics. 
\begin{marginfigure}
\includegraphics[width=5cm]{auxi/filament.jpg}
\caption{The material for the filament tungsten which can endure high temperature}
\end{marginfigure}

\begin{figure}[h]
\centering
\includegraphics[width=\linewidth]{auxi/IV.pdf}
\end{figure}

The characteristics of filament lamp are summarized as below:
\begin{SummBox}
\begin{itemize}
  \item Current \uline{(can/can not)} passes freely through either side of the lamp
  \item The filament lamp is \uline{(ohmic/non-ohmic)}
  \item For very small currents and voltages, the graph is roughly a straight line.
  \item As the voltage increases, the resistance of the filament lamp will \uline{(increase/decrease)}
\end{itemize}
\end{SummBox}

\begin{TaskBox}
Sketch the $I$-$V$ characteristic of filament lamp;
\tcblower
Explain why the resistance of filament lamp change under different voltage (Hint: connect it with the effect of temperature on resistance)
\end{TaskBox}

\section{Diodes}
Diodes are are made of \emph{semi-conductors}, which is also a kind of non-ohmic components. Watch the \href{https://www.bilibili.com/video/BV1FX4y1T7Nz}{video}. The most important thing about diodes is that it can only allows currents to pass through in only one direction.
\begin{marginfigure}
\includegraphics[width=4cm]{auxi/diodes.png}
\caption{Different types of diodes, LED is also a kind of diodes}
\end{marginfigure}

Thus the $I$-$V$ characteristic of diodes is as following:
\begin{figure}[h]
\centering
\includegraphics[width=\linewidth]{auxi/IV.pdf}
\end{figure}

The characteristics are summarised below:
\begin{SummBox}
\begin{itemize}
  \item If the p.d.(voltage) is not consistant with the direction the diode, no current flows in the diode
  \item When the p.d.(voltage) is consistant, if the p.d. does not exceed the \emph{threshold voltage}, there is still no current flowing in the diodes
  \item After the voltage is big enough, the diodes behaves like a ohmic resistor
\end{itemize}
\end{SummBox}

\section{NTC Thermistors}
Thermistors (thermal resistors) are components that are designed to have a \uline{resistance that changes rapidly with temperature}. There are two types of thermistors, NTC and PTC.
\begin{TaskBox}
Define NTC and PTC thermistors.
\vspace{0.5in}
\end{TaskBox}
However, when talking about thermistors, we are actually referring to the NTC thermistors. 
\begin{figure}[h]
\centering
\includegraphics[width=0.8\linewidth]{auxi/thermistor.jpg}
\caption{NTC is usally black packaged, PTC is usually blue packaged}
\end{figure}

The resistance characteristics are shown below:
\begin{figure}[h]
\centering
\includegraphics[width=0.6\linewidth]{auxi/thermistorcurve.png}
\caption{NTC's resistance is dependent of the temperature}
\end{figure}

\section{LDR}
A light-dependent resistor (LDR) is made of a high-resistance semiconductor, the resistance of which can vary greatly according to the light intensity\footnote{The unit for light intensity is \si{\candela}, which is also a base unit in SI system}.
\begin{SummBox}
Usually, if more light are cast on the LDR, the resistance will decrease.
\end{SummBox}
The variation of the resistance of a typical LDR with light intensity is shown below:
\begin{figure}[h]
\centering
\includegraphics[width=0.8\linewidth]{auxi/LDR.png}
\caption{A typical LDR}
\end{figure}

\end{document}
