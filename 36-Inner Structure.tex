\documentclass[a4paper]{tufte-handout}
% for debugging purposes -- displays the margins
%\geometry{showframe}
%\usepackage{float} %不知道和什么冲突了。
\usepackage{amsmath,amsfonts,amssymb,mathtools}
\newcommand{\icol}[1]{% inline column vector
  \left(\begin{smallmatrix}#1\end{smallmatrix}\right)%
}


\newcommand{\irow}[2]{ % 行向量,输出 xi+yj+zk的形式,存在的问题是不能自动根据负数调整为-号
  #1\mathbf{i}+#2\mathbf{j}%+#3\mathbf{k}
}
% \usepackage{geometry} %这个包已经在handout.cls使用过了,如果再调用一次会出问题
% \geometry{left=2cm,right=2cm,top=3cm,bottom=3cm}

%\usepackage[LGR]{fontenc} %使用希腊字符。

\usepackage{siunitx}
\DeclareSIUnit\torr{torr} %声明新的单位torr
\usepackage{mhchem}

\usepackage{BOONDOX-cal} %为了花体的emf符号
\usepackage[normalem]{ulem} %下划线
\usepackage{wrapfig}
\usepackage{caption}
\usepackage{float}

%定理的运用
\usepackage[english]{babel}
\newtheorem{theorem}{Theorem}[section]      %定理
\newtheorem{corollary}{Corollary}[theorem]  %结论
\newtheorem{lemma}[theorem]{Lemma}          %引理
\newtheorem{definition}[theorem]{Definition}%定义
% ------------------------------------------------------------------------------
% load hyperref to use hyperlinks
% ------------------------------------------------------------------------------
\usepackage{xcolor}
\definecolor{r1}{HTML}{FF8674}
\definecolor{b1}{HTML}{17ABDD}
\definecolor{p1}{HTML}{D4B6D6}
\definecolor{g1}{HTML}{70E2CB}
\definecolor{o1}{HTML}{DFA743}


\usepackage{hyperref}
\hypersetup{
      colorlinks=true,
      linkcolor=black,
      filecolor=cyan,
      urlcolor=b1,
      citecolor=g1,
}
% Set up the images/graphics package
\usepackage{graphicx}
%\graphicspath{{./auxi/}}
\setkeys{Gin}{width=0.7\linewidth,totalheight=0.3\textheight,keepaspectratio}

% The following package makes prettier tables.  We're all about the bling!
\usepackage{booktabs}

% The units package provides nice, non-stacked fractions and better spacing
% for units.
\usepackage{units}
\usepackage{nicefrac} %比较好看的斜体分号
\usepackage{siunitx}
\sisetup{separate-uncertainty}%产生的效果是是20+3cm 这样
% The fancyvrb package lets us customize the formatting of verbatim
% environments.  We use a slightly smaller font.
\usepackage{fancyvrb}
\fvset{fontsize=\normalsize}

% Small sections of multiple columns
\usepackage{multicol}

% Better variable font
\usepackage{mathptmx}

% todolist
\usepackage{enumitem}
\newlist{todolist}{itemize}{2}
\setlist[todolist]{label=$\square$} %因为美元符号的问题。需要手动添加美元\square美元

\usepackage{tikz}
\usepackage{circuitikz}

%beautifulbox
\usepackage{tcolorbox} %带背景色的盒子用于放置Summary,Task,还有Practice
\tcbuselibrary{breakable}
\tcbset{width=\textwidth} %默认盒子的宽度
\newenvironment{TaskBox} %任务盒子
{\begin{tcolorbox}[breakable,colback=b1!30,colframe=b1,title=Task]} {\end{tcolorbox}}
\newenvironment{ExampleBox} %Practice盒子
{\begin{tcolorbox}[breakable,colback=g1!30,colframe=g1,title=Example]} {\end{tcolorbox}}
\newenvironment{SummBox}
{\begin{tcolorbox}[breakable,colback=r1!30,colframe=r1,title=Summary]} {\end{tcolorbox}}


% These commands are used to pretty-print LaTeX commands
%\newcommand{\doccmd}[1]{\texttt{\textbackslash#1}}% command name -- adds backslash automatically
%\newcommand{\docopt}[1]{\ensuremath{\langle}\textrm{\textit{#1}}\ensuremath{\rangle}}% optional command argument
%\newcommand{\docarg}[1]{\textrm{\textit{#1}}}% (required) command argument
%\newenvironment{docspec}{\begin{quote}\noindent}{\end{quote}}% command specification environment
%\newcommand{\docenv}[1]{\textsf{#1}}% environment name
%\newcommand{\docpkg}[1]{\texttt{#1}}% package name
%\newcommand{\doccls}[1]{\texttt{#1}}% document class name
%\newcommand{\docclsopt}[1]{\texttt{#1}}% document class option name

\def\d{{\mathrm{d}}}

% 强制所有段落不缩进
\setlength{\parindent}{0pt}%没有起作用,日
\title{Inner Structure of Atoms}
\author{Sanjin Zhao}
\date{22th Nov, 2022}  % if the \date{} command is left out, the current date will be used


\begin{document}
\maketitle% this prints the handout title, author, and date
%\printclassoptions
\section*{Learning Outcome}
I highly recommend you to finish this checklist to determine whether you've achieved the learning objectives.
\begin{todolist}
  \item describe the nuclear model of the atom and the evidence for it
  \item understand and explain the $\alpha$ particle scattering experiment
  \item understand and recall the properties of proton, neutron, and nucleon
  \item recognize and label isotopes of same elements
\end{todolist}
\clearpage

\section{Leadin}
\begin{marginfigure}
\centering
\includegraphics[width=5cm]{auxi/plumpuddingmodel.png}
\caption{electrons resemble the plum in the cake, while the remaining postive part is filled with positve-charged body}
\end{marginfigure}
As we talked before, \href{https://www.britannica.com/biography/J-J-Thomson}{J.J. Thomson}'s cathode ray tube has unveiled the mask of \emph{eletrons} and hence plum-pudding model\footnote{Such model is actually proposed by William Thomson, Baron Kelvin of Largs. Not J.J. Thomson} was put forward.

Yet, this model has been refuted 40 years later by the Thomson's follower-Ernest Rutherford.
\begin{figure}[h]
\centering
\includegraphics[width=0.8\linewidth]{auxi/CathodeRayTube.png}
\caption{The cathode ray tube invented by Thomson}
\end{figure}

\section{$\alpha$-scattering}
Ernest Rutherford works collabratively with Thomson in Cavendish Laboratory\footnote{Because he talked so loud, there is a joking sign aimed at him}, and his famous alpha scattering has led people to another step further in investigating the inner structure.
\begin{marginfigure}
\centering
\includegraphics[width=3cm]{auxi/ThomsonRutherford.jpg}
\caption{Rutherford is the student of Thomson}
\end{marginfigure}

\subsection{Expriment setup}
The appratus is arranged as in Fig.\ref{fig:alpha scattering apparatus}. 
\begin{marginfigure}
\centering
\includegraphics[width=5cm]{auxi/alphasideview.jpg}
\caption{the side view for the apparatus}
\label{fig:alpha scattering apparatus}
\end{marginfigure}

The microscope can be moved in a circular track to detect the alpha participles from any directions\footnote{sometimes, a circular screen replaces the circular track}. You can try it in the \href{https://phet.colorado.edu/sims/html/rutherford-scattering/latest/rutherford-scattering_en.html}{Phet-Rutherford scattering}

\subsection{Result}
The result of the so called gold-foil experiment is quite impressive. It is shown as below:
\begin{figure}[h]
\centering
\includegraphics[width=0.95\linewidth]{auxi/Gold-Foil-Experiment.jpg}
\caption{The result of Rutherford Scattering}
\label{fig:result}
\end{figure}

\begin{SummBox}
You shall memorize the following famous phenomenon after the alpha particles is fired against the gold foil
\begin{itemize}
  \setlength\itemsep{1em}
  \item Most of the $\alpha$-particles \uline{\phantom{went through the foil without any deflection}}
  \item Some of the $\alpha$-particles \uline{\phantom{were deflected}} slightly
  \item Few of the $\alpha$-particles, but always be, does \uline{\phantom{bounce}} back, or the deflection angle is larger than \uline{\hspace{5em}}.
\end{itemize}
\end{SummBox}

Which observation do you think is the most striking result?

In Rutherford's point of view, he assimilates the result with the following quote, `It was almost as incredible as if you fired a 15-inch shell at a piece of tissue paper and it came back and hit you.'

\subsection{Explanation}
Despite his quote, Rutherford was not as surprised as he meant to be, since the experiment was in fact designed to prove his theory that \emph{the atom is almost empty!}. Such result was can be perfectly explained by the \emph{Rutherford-Bohr Model}\footnote{a.k.a nuclear model of the atom}.
\begin{marginfigure}
\centering
\includegraphics[width=5cm]{auxi/explanation.jpg}
\caption{possible result of $\alpha$ particles when passing the gold foil}
\end{marginfigure}

\begin{SummBox}
Explain each phenomenon listed above.
\vspace{3in}
\end{SummBox}


\section{Atomic Model}
Thomson has chopped the electrons, which carry negative charges, from the atoms; Rutherford has taken one step further, proposing that there is a tiny \emph{nucleus} that carries positive charge. And he futher proved that \textbf{proton} is present in all nuclei\footnote{plural form of nucleus}of any element in 1917.

Fifteen years later,Sir James Chadwick\footnote{won Noble Prize in 1935} discovered \textbf{neutron} by firing alpha particles to Beryllium. There is nothing mysterious about the nuclues.

The prevailing model of atom can be introduced now:
\begin{figure}[h]
\centering
\includegraphics[width=0.8\linewidth]{auxi/atomicmodel.jpg}
\caption{A simple model of atom, figure not drawn to scale}
\end{figure}

\subsection{Nucleon}
Nucleons refers to \uline{\hspace{1in}} and \uline{\hspace{1in}} in the nuclei.

\subsection{Property of particles}
Please finish the following tables of general information of tiny particles that consist an atom.
\begin{table}[h]
\begin{tabular}{|l|l|l|l|l|}
\hline
name    & symbol & mass/\si{\kg} & charge/\si{\coulomb} &charge/in $e$\\ \hline
proton   &        &      &       & \\ \hline
neutron  &        &      &       & \\ \hline
electron &        &      &       & \\ \hline
\end{tabular}
\end{table}

\begin{TaskBox}
The relative atomic mass is commonly used to desribe the mass of tiny atoms or particles. The unit is called `amu'-atomic mass unit, which is the mass of 1/12 of mass of the Carbon-12 atom. Thus, $\SI{1}{\amu}=\SI{1.66054e-27}{\kg}$.

State the relative atomic unit of proton and neutron in \si{\amu}.
\vspace{1in}
\end{TaskBox}

\section{Notation System of Atoms}
\subsection{Isotopes}
Playing with the PhET simulation \href{https://phet.colorado.edu/sims/html/build-a-nucleus/latest/build-a-nucleus_en.html}{Build nucleus}. You might find that despite two nuclei with same number of protons but different number of neutrons can both be stable\footnote{Which means that they can be found in nature}. Besides that, the element name is exactly the same. So such atoms with the nuclei are said to be \emph{isotopes}\footnote{Firstly coined by Dr Margaret Todd}of same element. 

\begin{TaskBox}
Define: isotope 
\vspace{1in}
\end{TaskBox}

\subsection{Labeling Isotopes}
In order to distinguish the isotopes/nuclides, standard notation system\footnote{a.k.a AZE notation} is introduced here as shown in Fig\ref{fig:aze notation}
\begin{marginfigure}
\centering
\includegraphics[width=5cm]{auxi/azenotation.png}
\caption{The AZE notation}
\label{fig:aze notation}
\end{marginfigure}
A represents the \uline{\hspace{1in}}\\
Z represents the \uline{\hspace{1in}}\\
E represents the \uline{\hspace{1in}}

Besides, due to the fact that, proton and neutron are basically euqal-weighted, and accounts for the most of the mass of an atom. the nucleon number Z can be also used as an proxy of the relative atomic mass.
\begin{figure}[h]
\centering
\includegraphics[width=0.8\linewidth]{auxi/peridoictable.png}
\caption{The relative atomic mass of gold-173 is appriximately equal to 173}
\label{fig:periodic table}
\end{figure}

\begin{TaskBox}
The Uranium-235 is not abundant in nature, but it is the fuel of atomic bombs. It has 143 nutrons. Write down the notation for this isotope
\end{TaskBox}
\end{document}

