\documentclass[a4paper]{tufte-handout}
% for debugging purposes -- displays the margins
%\geometry{showframe}
%\usepackage{float} %不知道和什么冲突了。
\usepackage{amsmath,amsfonts,amssymb,mathtools}
\newcommand{\icol}[1]{% inline column vector
  \left(\begin{smallmatrix}#1\end{smallmatrix}\right)%
}
\newcommand{\irow}[2]{ % 行向量,输出 xi+yj+zk的形式,存在的问题是不能自动根据负数调整为-号
  #1\mathbf{i}+#2\mathbf{j}%+#3\mathbf{k}
}
% \usepackage{geometry}
% \geometry{left=2cm,right=2cm,top=3cm,bottom=3cm}
\usepackage{siunitx}
\usepackage[normalem]{ulem}
\usepackage{wrapfig}
\usepackage{caption}
\usepackage{float}

%定理的运用
\usepackage[english]{babel}
\newtheorem{theorem}{Theorem}[section]      %定理
\newtheorem{corollary}{Corollary}[theorem]  %结论
\newtheorem{lemma}[theorem]{Lemma}          %引理
\newtheorem{definition}[theorem]{Definition}%定义
% ------------------------------------------------------------------------------
% load hyperref to use hyperlinks
% ------------------------------------------------------------------------------
\usepackage{xcolor}
\definecolor{r1}{HTML}{FF8674}
\definecolor{b1}{HTML}{17ABDD}
\definecolor{p1}{HTML}{D4B6D6}
\definecolor{g1}{HTML}{70E2CB}
\definecolor{o1}{HTML}{DFA743}

\usepackage{hyperref}
\hypersetup{
      colorlinks=true,
      linkcolor=black,
      filecolor=cyan,
      urlcolor=b1,
      citecolor=green,
}
% Set up the images/graphics package
\usepackage{graphicx}
\graphicspath{{./auxi/}}
\setkeys{Gin}{width=0.7\linewidth,totalheight=0.3\textheight,keepaspectratio}

\title{Terminal Velocity}
\author{Sanjin Zhao}
\date{12th Sep, 2022}  % if the \date{} command is left out, the current date will be used

% The following package makes prettier tables.  We're all about the bling!
\usepackage{booktabs}

% The units package provides nice, non-stacked fractions and better spacing
% for units.
\usepackage{units}
\usepackage{nicefrac} %比较好看的斜体分号
\usepackage{siunitx}
\sisetup{separate-uncertainty}%产生的效果是是20+3cm 这样
% The fancyvrb package lets us customize the formatting of verbatim
% environments.  We use a slightly smaller font.
\usepackage{fancyvrb}
\fvset{fontsize=\normalsize}

% Small sections of multiple columns
\usepackage{multicol}

% Better variable font
\usepackage{mathptmx}

% todolist
\usepackage{enumitem}
\newlist{todolist}{itemize}{2}
\setlist[todolist]{label=$\square$} %因为美元符号的问题。需要手动添加美元\square美元

\usepackage{tikz}

%beautifulbox
\usepackage{tcolorbox} %带背景色的盒子用于放置Summary,Task,还有Practice
\tcbuselibrary{breakable}
\tcbset{width=\textwidth} %默认盒子的宽度
\newenvironment{TaskBox} %任务盒子
{\begin{tcolorbox}[breakable,colback=b1!30,colframe=b1,title=Task]} {\end{tcolorbox}}
\newenvironment{ExampleBox} %Practice盒子
{\begin{tcolorbox}[breakable,colback=g1!30,colframe=g1,title=Example]} {\end{tcolorbox}}
\newenvironment{SummBox}
{\begin{tcolorbox}[breakable,colback=r1!30,colframe=r1,title=Summary]} {\end{tcolorbox}}


% These commands are used to pretty-print LaTeX commands
%\newcommand{\doccmd}[1]{\texttt{\textbackslash#1}}% command name -- adds backslash automatically
%\newcommand{\docopt}[1]{\ensuremath{\langle}\textrm{\textit{#1}}\ensuremath{\rangle}}% optional command argument
%\newcommand{\docarg}[1]{\textrm{\textit{#1}}}% (required) command argument
%\newenvironment{docspec}{\begin{quote}\noindent}{\end{quote}}% command specification environment
%\newcommand{\docenv}[1]{\textsf{#1}}% environment name
%\newcommand{\docpkg}[1]{\texttt{#1}}% package name
%\newcommand{\doccls}[1]{\texttt{#1}}% document class name
%\newcommand{\docclsopt}[1]{\texttt{#1}}% document class option name

\def\d{{\mathrm{d}}}

% 强制所有段落不缩进
\setlength{\parindent}{0pt}%没有起作用,日

\begin{document}
\maketitle% this prints the handout title, author, and date
%\printclassoptions
\section*{Learning Outcome}
I highly recommend you to finish this checklist to determine whether you've achieved the learning objectives.
\begin{todolist}
  \item Understand the effect of drag force or air resistance
  \item Define terminal velocity
  \item Draw and interpret the $v$-$t$ graph of object moving through air or liquids with or without \emph{parachute}
  \item Analyse the object's force in three stages, including accelerating, decelerating and terminal velocity.
\end{todolist}
\clearpage

\section{Leadin}
The raindrops usually falls from a height of \SI{1000}{\m} or even higher \footnote{\SI{2500}{\m} at most.}, according to the equation of free fall, the veloctiy that a raindrop hits the ground might be \SI{140}{\m\per\s} according to $v=\sqrt{2gh}$, nearly the speed of a bullet from a shotgun. But why nobody get hurt when a raindrop hit him/her? why

\section{Air resistance and Drag}
The answer to the question in the leadin part is quite simple, the speed can not actually reach that large. Because in real life, the air resistance can not be ignored. it will significantly decrease the final velocity. Usually the speed is around \SIrange{10}{30}{\m\per\s}. That's the effecet of air resistance or friction. Which should be taken into consideration when designing vehicles or high speed trains.
\begin{marginfigure}
\includegraphics[width=5cm]{aerodynamics.jpg}
\caption{wind tunnle test}
\end{marginfigure}

Anything moving in the fluids (including gas and liquid) will experiece resistive force called \emph{drag}. The formula for determine magnitude of drag would not be covered but factors that can influence the magnitude are listed below:
\begin{itemize}
  \item The \uline{\hspace{0.5in}} of the object
  \item The \uline{\hspace{0.5in}} of the object
  \item The \uline{\hspace{1in}} of the fluids
\end{itemize}


\section{Terminal Velocity}
Let's discuss the process of falling of a raindrop. When it first forms and drops, it is only subject to the weight, thus the acceleration of the raindrop \uline{euqals to the gravity}, causing the raindrop to accelerates downwards with an acceleration of $g$. 
\begin{marginfigure}
\centering
\includegraphics[width=3.5cm]{rainfall.jpg}
\end{marginfigure}
However, as the raindrop is gaining velocity, the air resistance will be directed upward and increase, impeding the falling of the raindrop. The resultant force on the raindrop is \uline{smaller than weight}, thus accleration is \uline{less than $g$}. But keep in mind, the velocity will still increase. Finally, the air resistance will completely \uline{cancel out graivty}, thus the resultant force on the raindrop will be \textbf{zero}. According to the NFL \footnote{Recall what is NFL}, the raindrop will now remain in uniform motion. That velocity is called \emph{Terminal Velocity}

\begin{TaskBox}
Based on the description, draw the $v$-$t$ graph of the raindrop falling with air reistance in the following graph.
\includegraphics[width=0.9\linewidth]{auxi/vtgraph.pdf}
\end{TaskBox}

\section{Fall with Parachute}
Skydivers will never imagine the day without their \emph{Parachute}.
\begin{marginfigure}
\includegraphics[width=4cm]{skydiver.jpg}
\caption{skydivers will always bring two parachutes}
\end{marginfigure}
The function of parachute is quite obvious - to minimize the speed when the skydiver lands. But the rational after it deserves a close check. 
\begin{figure}
\includegraphics[width=0.8\linewidth]{vtgraphofskydiver.jpg}
\caption{$v$-$t$ graph of a skydiver}
\label{fig:vtgraph of skydiver}
\end{figure}

\begin{TaskBox}
Using figure \ref{fig:vtgraph of skydiver} to explain:\\
(a). Why two terminal velocity seems to exist\\
(b). The resultant forces on the skydiver in four stages: at start, first terminal velocity, launch the parachute, and second terminal velocity.\\
\vspace{0.5 in}
\end{TaskBox}

\end{document}