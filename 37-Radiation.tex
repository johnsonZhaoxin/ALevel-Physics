\documentclass[a4paper]{tufte-handout}
% for debugging purposes -- displays the margins
%\geometry{showframe}
%\usepackage{float} %不知道和什么冲突了。
\usepackage{amsmath,amsfonts,amssymb,mathtools}
\newcommand{\icol}[1]{% inline column vector
  \left(\begin{smallmatrix}#1\end{smallmatrix}\right)%
}


\newcommand{\irow}[2]{ % 行向量,输出 xi+yj+zk的形式,存在的问题是不能自动根据负数调整为-号
  #1\mathbf{i}+#2\mathbf{j}%+#3\mathbf{k}
}
% \usepackage{geometry} %这个包已经在handout.cls使用过了,如果再调用一次会出问题
% \geometry{left=2cm,right=2cm,top=3cm,bottom=3cm}

%\usepackage[LGR]{fontenc} %使用希腊字符。

\usepackage{siunitx}
\DeclareSIUnit\torr{torr} %声明新的单位torr
\usepackage[version=4]{mhchem}

\usepackage{BOONDOX-cal} %为了花体的emf符号
\usepackage[normalem]{ulem} %下划线
\usepackage{wrapfig}
\usepackage{caption}
\usepackage{float}

%定理的运用
\usepackage[english]{babel}
\newtheorem{theorem}{Theorem}[section]      %定理
\newtheorem{corollary}{Corollary}[theorem]  %结论
\newtheorem{lemma}[theorem]{Lemma}          %引理
\newtheorem{definition}[theorem]{Definition}%定义
% ------------------------------------------------------------------------------
% load hyperref to use hyperlinks
% ------------------------------------------------------------------------------
\usepackage{xcolor}
\definecolor{r1}{HTML}{FF8674}
\definecolor{b1}{HTML}{17ABDD}
\definecolor{p1}{HTML}{D4B6D6}
\definecolor{g1}{HTML}{70E2CB}
\definecolor{o1}{HTML}{DFA743}


\usepackage{hyperref}
\hypersetup{
      colorlinks=true,
      linkcolor=black,
      filecolor=cyan,
      urlcolor=b1,
      citecolor=g1,
}
% Set up the images/graphics package
\usepackage{graphicx}
%\graphicspath{{./auxi/}}
\setkeys{Gin}{width=0.7\linewidth,totalheight=0.3\textheight,keepaspectratio}

% The following package makes prettier tables.  We're all about the bling!
\usepackage{booktabs}

% The units package provides nice, non-stacked fractions and better spacing
% for units.
\usepackage{units}
\usepackage{nicefrac} %比较好看的斜体分号
\usepackage{siunitx}
\sisetup{separate-uncertainty}%产生的效果是是20+3cm 这样
% The fancyvrb package lets us customize the formatting of verbatim
% environments.  We use a slightly smaller font.
\usepackage{fancyvrb}
\fvset{fontsize=\normalsize}

% Small sections of multiple columns
\usepackage{multicol}

% Better variable font
\usepackage{mathptmx}

% todolist
\usepackage{enumitem}
\newlist{todolist}{itemize}{2}
\setlist[todolist]{label=$\square$} %因为美元符号的问题。需要手动添加美元\square美元

\usepackage{tikz}
\usepackage{circuitikz}

%beautifulbox
\usepackage{tcolorbox} %带背景色的盒子用于放置Summary,Task,还有Practice
\tcbuselibrary{breakable}
\tcbset{width=\textwidth} %默认盒子的宽度
\newenvironment{TaskBox} %任务盒子
{\begin{tcolorbox}[breakable,colback=b1!30,colframe=b1,title=Task]} {\end{tcolorbox}}
\newenvironment{ExampleBox} %Practice盒子
{\begin{tcolorbox}[breakable,colback=g1!30,colframe=g1,title=Example]} {\end{tcolorbox}}
\newenvironment{SummBox}
{\begin{tcolorbox}[breakable,colback=r1!30,colframe=r1,title=Summary]} {\end{tcolorbox}}


% These commands are used to pretty-print LaTeX commands
%\newcommand{\doccmd}[1]{\texttt{\textbackslash#1}}% command name -- adds backslash automatically
%\newcommand{\docopt}[1]{\ensuremath{\langle}\textrm{\textit{#1}}\ensuremath{\rangle}}% optional command argument
%\newcommand{\docarg}[1]{\textrm{\textit{#1}}}% (required) command argument
%\newenvironment{docspec}{\begin{quote}\noindent}{\end{quote}}% command specification environment
%\newcommand{\docenv}[1]{\textsf{#1}}% environment name
%\newcommand{\docpkg}[1]{\texttt{#1}}% package name
%\newcommand{\doccls}[1]{\texttt{#1}}% document class name
%\newcommand{\docclsopt}[1]{\texttt{#1}}% document class option name

\def\d{{\mathrm{d}}}

% 强制所有段落不缩进
\setlength{\parindent}{0pt}%没有起作用,日
\title{Radiation}
\author{Sanjin Zhao}
\date{22th Nov, 2022}  % if the \date{} command is left out, the current date will be used


\begin{document}
\maketitle% this prints the handout title, author, and date
%\printclassoptions
\section*{Learning Outcome}
I highly recommend you to finish this checklist to determine whether you've achieved the learning objectives.
\begin{todolist}
  \item understand and interpret the standard notation of nuclide
  \item understand the unified atomic mass unit
  \item recall that strong nuclear force exists in nuclei
  \item understand that some isotopes of elements are not stable and will undergo radioactive decay
  \item show an understanding of the nature and properties of $\alpha$-, $\beta$- and $\gamma$- decay
  \item understand that in $\alpha$- and $\beta$- decay, a nuclide will changes into a different nuclide
  \item write and interpret the nuclear reaction equations
\end{todolist}
\clearpage

\section{Leadin}
Everything starts from the $\alpha-$ particle that Rutherford has used to explore the inner structure. \href{https://www.britannica.com/science/alpha-particle}{$\alpha$ particle} is in essece a helium neclues. How does it come from, what is inside the source that emit $\alpha$ particle. 
\begin{marginfigure}
\centering
\includegraphics[width=5cm]{auxi/alphaparticle.png}
\caption{$\alpha$ particle is emitted}
\label{fig:alpha particle}
\end{marginfigure}

\section{amu}
As a matter of fact, single atom has quite small mass, let alone the \emph{protons}, \emph{nutrons} and \emph{electrons} which is consist the atoms. 

In order to measure the tiny particles\footnote{definitely not in \si{\kg}}, the \textbf{Unfied Atomic Mass Unit}-amu is adopted.
\begin{SummBox}
Define what is amu
\vspace{0.5in} 
\end{SummBox}

So the basic principle is that dividing the mass of any atoms by the mass,in \si{\kg}, of amu. Then, you will get the \uline{\hspace{1in}} of the atoms\footnote{It is quite common that the value is not an integer.}.

\begin{table}[h]
\begin{tabular}{|l|l|l|l|l|}
\hline
name    & symbol & mass/\si{\kg} & mass/\si{\amu} \\ \hline
proton   &        &      &        \\ \hline
neutron  &        &      &        \\ \hline
electron &        &      &        \\ \hline
\end{tabular}
\end{table}

Here, this is the interesting appears.
Calculate the total sum of 6 protons and 6 neutrons and 6 electrons, The result is:\uline{\hspace{1in}}. You might find it does not equal to 12, which by definition should be the relative atomic mass of carbon-12. Why is that\footnote{Have you ever got any ideas, if so please write down your guess in the margin}?! If you are interested in that, refer to Einstein's \href{https://www.atomicarchive.com/science/physics/einsteins-equation.html}{mass-energy equation}. 


\section{Strong Nuclear Force}
As previously described, \textcolor{r1}{Strong Nuclear Force} or \textbf{Strong Interaction} acts between the \uline{\hspace{1in}} and \uline{\hspace{1in}}, which can hold nucleus together tightly.

Strong Nuclear Force is one the four \textbf{fundamental forces}. It will only act over short distance (approximately \SIrange{1e-15}{1e-14}{\m})

\subsection{Unstability in larger nuclei}
Protons are born to be repulsive when they are close to each other, this force is called coloumb force or \textbf{Electromagnetism force}. Neutrons, in such cirumstance, could be used to be an `glue' to seperate the protons and make the nuclei possible.

However, different isotopes will show different stabilities due to the competition between the strong interaction and the electromagnetism replusion\footnote{This is a simple explanation, not quite accurate}. Hence, a lot of isotopes will undergo \textbf{radioactive decay} to transform into a more stable states.

\section{Radioactive Decay}
In this section, three types of decays is introduced, they are $\alpha$-, $\beta$- and $\gamma$-decay. 

\subsection{Henri Becquerel}
\href{https://www.britannica.com/biography/Henri-Becquerel}{Henri Becquerel} shared the Nobel Prize with Pierre and Marie Curie for their discovery of radioactivity. For his contribution, his name become lowercased and turned into the unit - `becquerel' which measure the radioactivity of substance. 
\begin{marginfigure}
\centering
\includegraphics[width=3cm]{auxi/becquerel.jpg}
\caption{Henri Becquerel 1852-1908}
\end{marginfigure}

\subsection{$\alpha$-decay}
\begin{marginfigure}
\includegraphics[width=4cm]{auxi/ionization.jpg}
\caption{$\alpha$ particle can cause ionization of atoms}
\end{marginfigure}

A nucleus in \ref{fig:alpha particle} undergoes $\alpha$-decay, and emits $\alpha$-particles. These $\alpha$-particle - in other form, the helium nucleus - will travel in relatively low speed and due to its large mass and size, energy of these particles will cause \textbf{ionization} of neighbouring atoms.
\begin{figure}[h]
\centering
\includegraphics[width=0.8\linewidth]{auxi/alphatrack.jpg}
\caption{the tracks of moving $\alpha$ particles in cloud chamber}
\end{figure}

\begin{TaskBox}
An nucleus emitting $\alpha$ particle will lose \uline{\hspace{4em}} protons and \uline{\hspace{4em}} neutrons and \uline{\hspace{4em}} electron(s). And due to the change of number of protons, the original nucleus will turn into a new nucleus of another element.
\end{TaskBox}

\subsection{$\beta$-decay}
Two types of $\beta$ particles will form when a nucleus undergo $\beta$-decay. $\beta^+$ and $\beta^-$ particle, which are actually \uline{\hspace{4em}} and \uline{positron} respectively\footnote{you can memorize positron as `posi'tive elec`tron'}. Positron is the \textbf{antimatter} of electron, which is made up of by antipartiples of that which electron include.

In $\beta^-$ decay, the \uline{neutron} in the nucleus will decay into a proton, and hence generating an electron. The electron, aka $\beta^-$ particle will run out of the nucleus. 

\begin{TaskBox}
Explain the $\beta^+$ decay:
\vspace{1.5in}
\end{TaskBox}

\begin{marginfigure}
\centering
\includegraphics[height=5cm]{auxi/betatrack.png}
\caption{the track of $\beta^-$ particle in cloud chamber}
\end{marginfigure}

One thing to mention is that, in $\beta$ decay, there exist another type of particle are emitted, but such particle barely interacts with matters at all, has little or no rest mass, no charge. It was not until 1956 
that such particle were discovered. The name for them are \href{https://www.scientificamerican.com/article/what-is-a-neutrino/}{neutrino}, we will discuss this more in particle physics.


\subsection{$\gamma$ decay}
You might noticed that we don't talk about $\gamma$ particles, since the result of $\gamma$ decay is not particle\footnote{wave-particle duality is not breakable} but a $\gamma$ ray.

Therefore, unlike $\alpha$ decay and $\beta$ decay, the nuclues does not change, it is the electron outside will release energy, transforming atoms at excited states to lower states. Those energy is transformed into the energy of $\gamma$ ray.

One thing to mention, usually $\gamma$ radiation can occur in company with $\beta$ radiation.
\begin{figure}[h]
\centering
\includegraphics[width=0.8\linewidth]{auxi/betagamma.jpg}
\end{figure}
Since it is `factotum' for the energy to be released.

\subsection{Penetrating Ability}
\begin{figure}[h]
\centering
\includegraphics[width=0.8\linewidth]{auxi/RadiationPenetration.jpg}
\end{figure}

Due the quick loss in energy, $\alpha$ particles does not show a strong penetration. Pieces of paper with thickness up to \uline{\hspace{1in}} would be enough to block the particles.

While, since $\beta^-$ and $\beta^+$ particles are small in mass, large in speed. An aluminum foil with thickness \uline{\hspace{1in}} will totally abosorb the radiation. 

For the $\gamma$ ray emitted from $\gamma$ radiation, since it is in the form of EM wave, and moving at the speed of light. To fully prevent the radiation. A \uline{lead} wall with thickness of \uline{\hspace{1in}} would be enough. 

\begin{figure}[h]
\centering
\includegraphics[width=0.8\linewidth]{auxi/hulk.jpg}
\caption{Nobody knows gamma ray better than me - Bruce Banner}
\end{figure}
Read the story of \href{https://www.marvel.com/characters/hulk-bruce-banner/on-screen/profile}{Bruce Banner} to discover the damage that $\gamma$ ray does to human body.

\section{Equation for Radiation}
With the standard notation, we can now transform the picture depiction of radiaton to symbol system 
\begin{figure}[h]
\centering
\includegraphics[width=0.8\linewidth]{auxi/Alpha-Decay.png}
\caption{$alpha$ decay of Uranium-238}
\label{fig:alpha decay of U238}
\end{figure}

In Fig.\ref{fig:alpha decay of U238}, the process can be expressed as following:
\begin{center}
  $\ce{^{238}_{92}U ->^{234}_{90}Th + ^{4}_{2}\alpha}$
\end{center}
or the $^{4}_{2}\alpha$ can be interchangeable as $\ce{^{4}_{2}He}$.

This equation shows what happens for the $\ce{^{238}_{92}U}$ nucleus, which is called the parent nucleus, and the $\ce{^{234}_{90}Th}$ is called the daughter nucleus.

\begin{TaskBox}
what is the relationship between the nucleon number and the atomic number of the parent and daughter nuclei?
\vspace{0.5in}
\end{TaskBox}

The equation for three types of radiation are summarised below:
\begin{figure}[h]
\centering
\includegraphics[width=0.8\textwidth]{auxi/radiationsummary.jpg}
\caption{The reactant and product of radiation}
\label{fig:summ}
\end{figure}

In Fig.\ref{fig:summ}, $\beta^+$ decay is not shown, please finish it. 
\end{document}
