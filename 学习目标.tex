\documentclass[a4paper]{tufte-handout}
% for debugging purposes -- displays the margins
%\geometry{showframe}
%\usepackage{float} %不知道和什么冲突了。
\usepackage{amsmath,amsfonts,amssymb,mathtools}
\newcommand{\icol}[1]{% inline column vector
  \left(\begin{smallmatrix}#1\end{smallmatrix}\right)%
}
\newcommand{\irow}[2]{ % 行向量,输出 xi+yj+zk的形式,存在的问题是不能自动根据负数调整为-号
  #1\mathbf{i}+#2\mathbf{j}%+#3\mathbf{k}
}
% \usepackage{geometry}
% \geometry{left=2cm,right=2cm,top=3cm,bottom=3cm}

\usepackage{siunitx}
\usepackage[normalem]{ulem}
\usepackage{wrapfig}
\usepackage{caption}
\usepackage{float}


% ------------------------------------------------------------------------------
% load hyperref to use hyperlinks
% ------------------------------------------------------------------------------
\usepackage{xcolor}
\definecolor{r1}{HTML}{FF8674}
\definecolor{b1}{HTML}{17ABDD}
\definecolor{p1}{HTML}{D4B6D6}
\definecolor{g1}{HTML}{70E2CB}
\definecolor{o1}{HTML}{DFA743}

\usepackage{hyperref}
\hypersetup{
      colorlinks=true,
      linkcolor=black,
      filecolor=cyan,
      urlcolor=b1,
      citecolor=green,
}



% Set up the images/graphics package
\usepackage{graphicx}
\setkeys{Gin}{width=0.7\linewidth,totalheight=0.3\textheight,keepaspectratio}
%\graphicspath{{./auxi/}}

\title{ALevel Physics}
\author{Sanjin Zhao}
\date{1st Jan 2023}  % if the \date{} command is left out, the current date will be used

% The following package makes prettier tables.  We're all about the bling!
\usepackage{booktabs}

% The units package provides nice, non-stacked fractions and better spacing
% for units.
\usepackage{units}

% 国际制单位,使用有点问题好像
\usepackage{siunitx}
\sisetup{separate-uncertainty}%
% The fancyvrb package lets us customize the formatting of verbatim
% environments.  We use a slightly smaller font.
\usepackage{fancyvrb}
\fvset{fontsize=\normalsize}

% Small sections of multiple columns
\usepackage{multicol}

% Better variable font
\usepackage{mathptmx}

% todolist
\usepackage{enumitem}
\newlist{todolist}{itemize}{2}
\setlist[todolist]{label=$\square$}

\usepackage{tikz}

%beautifulbox
\usepackage{tcolorbox} %带背景色的盒子用于放置Summary,Task,还有Practice
\tcbuselibrary{breakable}
\tcbset{width=\textwidth} %默认盒子的宽度
\newenvironment{TaskBox} %任务盒子
{\begin{tcolorbox}[breakable,colback=b1!30,colframe=b1,title=Task]} {\end{tcolorbox}}
\newenvironment{ExampleBox} %Practice盒子
{\begin{tcolorbox}[breakable,colback=g1!30,colframe=g1,title=Example]} {\end{tcolorbox}}
\newenvironment{SummBox}
{\begin{tcolorbox}[breakable,colback=r1!30,colframe=r1,title=Summary]} {\end{tcolorbox}}


% These commands are used to pretty-print LaTeX commands
\newcommand{\doccmd}[1]{\texttt{\textbackslash#1}}% command name -- adds backslash automatically
\newcommand{\docopt}[1]{\ensuremath{\langle}\textrm{\textit{#1}}\ensuremath{\rangle}}% optional command argument
\newcommand{\docarg}[1]{\textrm{\textit{#1}}}% (required) command argument
\newenvironment{docspec}{\begin{quote}\noindent}{\end{quote}}% command specification environment
\newcommand{\docenv}[1]{\textsf{#1}}% environment name
\newcommand{\docpkg}[1]{\texttt{#1}}% package name
\newcommand{\doccls}[1]{\texttt{#1}}% document class name
\newcommand{\docclsopt}[1]{\texttt{#1}}% document class option name


\begin{document}

\maketitle% this prints the handout title, author, and date

%\printclassoptions

\section{SI system}
\begin{todolist}
  \item Understand that all physical quantities consist of a numerical magnitude and a unit.
  \item Recall the following SI \textbf{base quantities} and their units:
  \item Recall and use \textbf{prefixes} and their symbols
  \item Make reasonable estimates of physical quantities included within the syllabus
  \item Express \textbf{derived units} as products or quotients of the SI base units
  \item Use SI base units to check the \emph{homogeneity}
  % \item Understand the difference between scalar and vector quantities
  % \item Add and subtract coplanar vectors
  % \item Represent a vector as two perpendicular components
\end{todolist}
\clearpage

\section{scalar v.s. vector}
\begin{todolist}
  \item Understand the difference between \emph{scalar and vector} quantities
  \item \textbf{Add} and \textbf{subtract} coplanar vectors
  \item Represent a vector as \textbf{two perpendicular components}
  \item Using coordinate expression to calculate the scalar product
\end{todolist}
\clearpage

\section{s.f.}
\begin{todolist}
  \item grasp \emph{Scientific Notation}
  \item read and count significant figures
  \item \emph{Mathematic Rules} for sig fig calculations
  \item do \emph{Rounding} of values
\end{todolist}
\clearpage

\section{Error and Uncertainty}
\begin{todolist}
  \item distinguish \textbf{true value} and readings.
  \item know the difference between \emph{accuracy} and \emph{precision}
  \item distinguish between \emph{Random Error} and \emph{Systematic Error}\footnote{def:}
  \item state methods to minimize such errors
  \item express uncertainty in both \emph{absolute} and \emph{relative} forms
  \item grasp the operation rules for uncertainty
\end{todolist}
\clearpage

\section{vernier and micrometer}
\begin{todolist}
	\item Learn to read \emph{verniers} and \emph{microscrew gauge}
	\item Know the \emph{absolute uncertaity} of the device
	\item Use verniers and microscrew gauge to \textbf{measure} tiny length
	\item Know the common \emph{measuring equipment} in the lab
\end{todolist}
\clearpage


\section{describe motion}
\begin{todolist}
  \item Define and use \emph{distance}, \emph{displacement}, \emph{speed}, \emph{velocity} and \emph{acceleration}.
  \item Describe labratory methods for determining the speed of of an object
  \item Use graphical methods to represent motional quantities, such as $d-t$ graph or $v-t$ graph
  \item Determine displacement from the \textbf{area} under a \emph{velocity–time graph}.
  \item Determine velocity using the \textbf{gradient} of a \emph{displacement–time graph}.
  \item Determine acceleration using the \textbf{gradient} of a \emph{velocity–time graph}.
\end{todolist}
\clearpage


\section{UAM}
\begin{todolist}
  \item Determine displacement from the \textbf{area} under a \emph{velocity–time graph}.
  \item Determine velocity using the \textbf{gradient} of a \emph{displacement–time graph}.
  \item Determine acceleration using the \textbf{gradient} of a \emph{velocity–time graph}.
  \item Derive, from the definitions of velocity and acceleration, equations that represent uniformly accelerated motion in a straight line.
  \item Solve problems using equations of uniformly accelerated motion. including free fall
  \item Draw $d$-$t$, $v$-$t$ and $a$-$t$ for stationary, uniform motion, uniformly acclerated motion, free fall or thrown up. 
  \item Describe an experiment to determine the acceleration of free fall using a falling object
  \item Describe and explain motion due to a uniform velocity in one direction and a uniform acceleration in a perpendicular direction
\end{todolist}
\clearpage

\section{free fall}
\begin{todolist}
  \item Recognize the kinematic characteristics of \emph{free fall}
  \item Draw $v$-$t$ graph of free fall or thrown up
  \item Deduce the equation of motion for the free falling object from the equation of motion of UAM
  \item Using equations to solve free fall problems
  \item Recognize the kinmatic characteristics of throwning upward
  \item Using equations to solve upward throwing problems
  % \item Recognize \emph{Projectile Motion}
  % \item Grasp the decomposition of projectile motion
  % \item Using equations to solve projectile motion
\end{todolist}
\clearpage


\section{Projectile}
\begin{todolist}
  \item Recognize the kinematic characteristics of \emph{horizontal projectile motion} or \emph{projectile motion} 
  \item Grasp the way to decompose vectors and analyze the \emph{Two Dimensional Motion}
  \item Using equation of free fall to solve problems related to horizontal projectile motion or projectile motion
\end{todolist}
\clearpage

\section{terminal velocity}
\begin{todolist}
  \item Understand the effect of drag force or air resistance
  \item Define terminal velocity
  \item Draw and interpret the $v$-$t$ graph of object moving through air or liquids with or without \emph{parachute}
  \item Analyse the object's force in three stages, including accelerating, decelerating and terminal velocity.
\end{todolist}
\clearpage

\section{forces}
\begin{todolist}
  \item Define and distinguish different types of forces
  \item Use a vector triangle to represent coplanar forces in equilibrium
  \item Add two or more coplanar forces
  \item Resolve a force into perpendicular components
  \item Recognise that mass is a property of an object that resists change in motion
  \item Recall that the weight of a body is equal to the product of its mass and the acceleration of free fall
  \item Represent the weight of a body as acting at a single point known as its centre of gravity
\end{todolist}
\clearpage


\section{moment and torque}
\begin{todolist}
  \item Define and apply the \emph{moment}\footnote{def:} of a force and the \emph{torque of a couple}\footnote{def:}
  \item State and apply the principle of moments
  \item Use the idea that, when there is no resultant force and no resultant torque, a system is in equilibrium.
  \item calculate the \emph{work done} by a torque or moment
\end{todolist}
\clearpage

\section{work done by a force}
\begin{todolist}
  \item Use the concept of work and energy
  \item Derive and use the formulae for \emph{kinetic energy}(k.e.) and \emph{gravitational potential energy}(g.p.e.).
  \item Recall and apply the \emph{principle of conservation of energy}\footnote{conservation is a quite important law in physics}
  %\item Recall and understand that the \emph{efficiency} of a system
  %\item Use the concept of efficiency to solve problems
\end{todolist}
\clearpage

\section{power and efficiency}
\begin{todolist}
  \item Define and use the equation for power using $P=\nicefrac{W}{t}$ and derive $P = Fv$
  \item Recall and understand that the \emph{efficiency} of a system
  \item Use the concept of efficiency to solve problems
\end{todolist}
\clearpage


\section{linear momentum}
\begin{todolist}
  \item Define and use \emph{linear momentum}
  \item Define and use \emph{impluse}\footnote{not required by CAIE but by me}
  \item Recall and use that the \textbf{area} under $F$-$t$ graph equals to the \textbf{change of momentum}
  \item Relate force to the rate of change of momentum and state Newton’s second law of motion
\end{todolist}
\clearpage

\section{collision}
\begin{todolist}
  \item state and apply the principle of conservation of momentum to collisions in one and two dimensions
  \item apply the principle of conservation of momentum to solve simple problems, including elastic and inelastic interactions between objects in both one and two dimensions\footnote{knowledge of the concept of \emph{coefficient of restitution} is not required. But Sanjin Zhao would be much happier if you dig into that.}
  \item solve the velocity after elastic and total inelastic collision
  \item recall that, for a perfectly elastic collision, the relative speed of approach is equal to the relative speed of separation
  \item discuss energy changes in perfectly elastic and inelastic collisions
\end{todolist}
\clearpage

\section{density and pressure}
\begin{todolist}
  \item define and use density
  \item define and use pressure and calculate the pressure in a fluid
  \item derive and use the equation: $\Delta p= \rho g h$
  \item explain and use Achimedes' principle
  \item use a difference in hydrostatic pressure to explain and calculate upthrust
\end{todolist}
\clearpage

\section{spring}
\begin{todolist}
  \item explain how tensile and compressive forces cause deformation
  \item describe the behavior of springs and use Hooke's Law\footnote{formula?}
  \item distinguish between elastic and plastic deformation, limit of proportionality and the elastic limit
  % \item define and use \emph{stress, strain} and the \emph{Young Modulus}
  % \item describe an experiment to measure the Young Modulus
  % \item calculate the energy stored in a deformed materials, from both Hooke's Law and Young Modulus perspective.
\end{todolist}
\clearpage

\section{Young Modulus}
\begin{todolist}
  \item define and use \emph{stress, strain} and the \emph{Young Modulus}
  \item describe an experiment to measure the Young Modulus
  \item \emph{compare} Young Modulus and spring constant of a wire
  \item calculate the energy stored in a deformed materials, from both Hooke's Law and Young Modulus perspective.
\end{todolist}
\clearpage

\section{charges*}
\begin{todolist}
  \item understand that atoms has negative electron surrouding
  \item explain the phenomenon of charge by friction
  \item use electroscope to test whether an object is charged or not
  \item understand that charge is quantised
  \item memorize the elementary charge
  \item understand the term charge and recognise its unit, the coulomb

  % \item understand of the nature of electric current

  % \item solve problems using the equation $Q = It$
  % \item solve problems using the formula $I = nAve$
  % \item solve problems involving the mean drift velocity of charge carriers
\end{todolist}
\clearpage

\section{current}
\begin{todolist}
  \item understand of the nature of electric current
  \item solve problems using the equation $Q = It$ or $I=\nicefrac{\Delta Q}{\Delta t}$
  \item solve problems using the formula $I = nqvA$ or $I=nAve$
  \item solve problems involving the mean drift velocity of charge carriers
\end{todolist}
\clearpage

\section{p.d.}
\begin{todolist}
  \item define the potential difference across a component as the energy transferred per unit charge
  \item understand the terms \emph{potential difference}, e.m.f. and the volt
  \item use energy considerations to distinguish between p.d. and e.m.f. ($\mathcal{E}$)
  \item recall and use $\Delta V=\nicefrac{W}{Q}$
  \item recall and use $P = VI$
\end{todolist}
\clearpage

\section{Ohm's Law}
\begin{todolist}
  \item state Ohm’s law
  \item sketch and explain the $I$–$V$ characteristics for resistors
  %\item sketch the temperature characteristic for an NTC thermistor
  \item solve problems involving the resistivity of a material.
\end{todolist}
\clearpage

\section{electric components}
\begin{todolist}
  \item understand that metals have \emph{delocalised electrons} to be conductive
  \item explain why temeperature could change the resistance, and the phenomenon of \emph{superconducitiviy}
  \item sketch and explain the $I$–$V$ characteristics for \emph{filament lamp}, \emph{diodes}, \emph{thermistor}
  \item sketch the temperature characteristic for an NTC thermistor
  \item sketch the light intensity characteristics for an light-dependent resistor(LDR)
\end{todolist}
\clearpage

\section{Kirchhoff's Law}
\begin{todolist}
  \item recall and apply \emph{Kirchhoff’s laws}, KCL and KVL both
  \item analyse the circuit using Kirchhoff's law
  \item use Kirchhoff’s laws to derive the formulae for the combined resistance of two or more resistors \emph{in series and in parallel}
  \item recognise that ammeters are connected in series within a circuit and therefore should have low resistance
  \item recognise that voltmeters are connected in parallel across a component, or components, and therefore should have high resistance.
\end{todolist}
\clearpage

\section{connection}
\begin{todolist}
  \item use Kirchhoff’s laws to derive the formulae for the combined resistance of two or more resistors \emph{in series and in parallel}
  \item recognise that ammeters are connected in series within a circuit and therefore should have low resistance
  \item recognise that voltmeters are connected in parallel across a component, or components, and therefore should have high resistance.
\end{todolist}
\clearpage

\section{practical circuit}
\begin{todolist}
  \item explain the effects of \emph{internal resistance} on \emph{terminal p.d.} and power output of a source of e.m.f.
  \item describe experiments to decide the e.m.f. and internal resistance of a source or battery
  \item explain the use of \emph{potential divider}\footnote{def:} circuits
  \item solve problems involving the potentiometeras a means of comparing voltage
\end{todolist}
\clearpage

\section{describe waves}
\begin{todolist}
  \item describe a progressive wave
  \item describe the motion of transverse and longitudinal waves
  \item describe waves in terms of their wavelength, amplitude, frequency, speed, phase difference and intensity
  \item use the time-base and y-gain of a cathode-ray oscilloscope (CRO) to determine frequency and amplitude
  \item use the wave equation $v = f\cdot \lambda$ and $f =\nicefrac{1}{T}$
  \item use the equations of intensity and its relationship with amplitude.
\end{todolist}
\clearpage

\section{Doppler Effect}
\begin{todolist}
  \item describe the \emph{Doppler Effect} for sound waves and electromagnetic waves, intensity and frequency.
  \item use the equation $f_0=\frac{f_s \cdot v}{v\pm v_s}$, know the conditions of when to use $+/-$
\end{todolist}
\clearpage


\section{EM waves}
\begin{todolist}
  \item describe and understand \emph{electromagnetic waves}
  \item recall that wavelengths in the range \SIrange{400}{700}{\nm} in free space are visible to the human eye
  \item describe and understand \emph{polarisation}
  \item use Malus’s law to determine the intensity of transmitted light through a polarising filter.
\end{todolist}
\clearpage


