\documentclass[a4paper]{tufte-handout}
% for debugging purposes -- displays the margins
%\geometry{showframe}
%\usepackage{float} %不知道和什么冲突了。
\usepackage{amsmath,amsfonts,amssymb,mathtools}
\newcommand{\icol}[1]{% inline column vector
  \left(\begin{smallmatrix}#1\end{smallmatrix}\right)%
}
\newcommand{\irow}[2]{ % 行向量,输出 xi+yj+zk的形式,存在的问题是不能自动根据负数调整为-号
  #1\mathbf{i}+#2\mathbf{j}%+#3\mathbf{k}
}
% \usepackage{geometry}
% \geometry{left=2cm,right=2cm,top=3cm,bottom=3cm}
\usepackage{siunitx}
\usepackage[normalem]{ulem}
\usepackage{wrapfig}
\usepackage{caption}
\usepackage{float}
% ------------------------------------------------------------------------------
% load hyperref to use hyperlinks
% ------------------------------------------------------------------------------
\usepackage{xcolor}
\definecolor{r1}{HTML}{FF8674}
\definecolor{b1}{HTML}{17ABDD}
\definecolor{p1}{HTML}{D4B6D6}
\definecolor{g1}{HTML}{70E2CB}
\definecolor{o1}{HTML}{DFA743}

\usepackage{hyperref}
\hypersetup{
      colorlinks=true,
      linkcolor=black,
      filecolor=cyan,
      urlcolor=b1,
      citecolor=green,
}
% Set up the images/graphics package
\usepackage{graphicx}
\graphicspath{{./auxi/}}
\setkeys{Gin}{width=0.7\linewidth,totalheight=0.3\textheight,keepaspectratio}

\title{Projectile Motion}
\author{Sanjin Zhao}
\date{5th Sep, 2022}  % if the \date{} command is left out, the current date will be used

% The following package makes prettier tables.  We're all about the bling!
\usepackage{booktabs}

% The units package provides nice, non-stacked fractions and better spacing
% for units.
\usepackage{units}
\usepackage{nicefrac} %比较好看的斜体分号
\usepackage{siunitx}
\sisetup{separate-uncertainty}%产生的效果是是20+3cm 这样
% The fancyvrb package lets us customize the formatting of verbatim
% environments.  We use a slightly smaller font.
\usepackage{fancyvrb}
\fvset{fontsize=\normalsize}

% Small sections of multiple columns
\usepackage{multicol}

% Better variable font
\usepackage{mathptmx}

% todolist
\usepackage{enumitem}
\newlist{todolist}{itemize}{2}
\setlist[todolist]{label=$\square$} %因为美元符号的问题。需要手动添加美元\square美元

\usepackage{tikz}

%beautifulbox
\usepackage{tcolorbox} %带背景色的盒子用于放置Summary,Task,还有Practice
\tcbuselibrary{breakable}
\tcbset{width=\textwidth} %默认盒子的宽度
\newenvironment{TaskBox} %任务盒子
{\begin{tcolorbox}[breakable,colback=b1!30,colframe=b1,title=Task]} {\end{tcolorbox}}
\newenvironment{ExampleBox} %Practice盒子
{\begin{tcolorbox}[breakable,colback=g1!30,colframe=g1,title=Example]} {\end{tcolorbox}}
\newenvironment{SummBox}
{\begin{tcolorbox}[breakable,colback=r1!30,colframe=r1,title=Summary]} {\end{tcolorbox}}


% These commands are used to pretty-print LaTeX commands
%\newcommand{\doccmd}[1]{\texttt{\textbackslash#1}}% command name -- adds backslash automatically
%\newcommand{\docopt}[1]{\ensuremath{\langle}\textrm{\textit{#1}}\ensuremath{\rangle}}% optional command argument
%\newcommand{\docarg}[1]{\textrm{\textit{#1}}}% (required) command argument
%\newenvironment{docspec}{\begin{quote}\noindent}{\end{quote}}% command specification environment
%\newcommand{\docenv}[1]{\textsf{#1}}% environment name
%\newcommand{\docpkg}[1]{\texttt{#1}}% package name
%\newcommand{\doccls}[1]{\texttt{#1}}% document class name
%\newcommand{\docclsopt}[1]{\texttt{#1}}% document class option name

\def\d{{\mathrm{d}}}

% 强制所有段落不缩进
\setlength{\parindent}{0pt}%没有起作用,日

\begin{document}
\maketitle% this prints the handout title, author, and date
%\printclassoptions
\section*{Learning Outcome}
I highly recommend you to finish this checklist to determine whether you've achieved the learning objectives.
\begin{todolist}
  \item Recognize the kinematic characteristics of \emph{horizontal projectile motion} or \emph{projectile motion} 
  \item Grasp the way to decompose vectors and analyze the \emph{Two Dimensional Motion}
  \item Using equation of free fall to solve problems related to horizontal projectile motion or projectile motion
\end{todolist}
\clearpage

\section{Leadin}
Life is so hard, how could every motion in the real world always be straight or linear? The investigate should be furthered, at least into two dimensional or \emph{planar motion}.
\begin{marginfigure}
\includegraphics[width=5cm]{Projectile-Motion.jpeg}
\caption{The multiflash photo of a motor racer}
\label{fig:motorcycle}
\end{marginfigure}

\section{Horizontal Projectile}
One of the most frequently studied two dimensional motion is \emph{projectile}\footnote{def:}, in which the object has an initial \textbf{velocity} and is subject only to \textbf{gravity}. The moving path is shown below:
\begin{figure}[h]
\includegraphics{horiproj.pdf}
\caption{The path of an object's \emph{horizontal projectile motion}}
\label{fig:parabola}
\end{figure}

You can click on this \href{https://phet.colorado.edu/sims/html/projectile-motion/latest/projectile-motion_en.html}{Phet} simulation to understand projectile better. 

\subsection{Decompose vectors}
Recall that the displacement and vectors are vectors. Which means it can be viewed as a combination of two perpendicular components.
\begin{TaskBox}
In figure \ref{fig:parabola}, label the displacement of the final position, and decompose it into horizontal displacement $\vec{x}$ and vertical displacement$\vec{y}$
\end{TaskBox}

Just as the displacement, the velocity at any position can be decomposed into horizontal and vertical components $\vec{v_x}$ and $v_y$. In figure \ref{fig:parabola}. The blue arrow represent the \uline{\hspace{0.5 in}} and the green arrow represent the \uline{\hspace{0.5 in}}. The black arrow is the resultant instantaneous and the true velocity of such motion. 
\begin{TaskBox}
\includegraphics[width=0.5\linewidth]{vector.pdf}\\
Try to decompose the velocity vector above, and using trigonometrical ratio to express the magnitude of horizontal and vertical component.
\vspace{0.5\in}
\end{TaskBox}

\subsection{Final Conclusion}
Check the \href{https://phet.colorado.edu/sims/html/projectile-motion/latest/projectile-motion_en.html}{Phet} simulation, and turn on the components visualizaiton? What do you find?
\begin{SummBox}
The horizontal velocity \uline{\hspace{2in}}\\
The vertical velocity \uline{\hfill}.\\
Thus:\\
If viewed only in the vertical direction, the object experiencing horizontal projectile will be exactly the same as another object which experience free fall.
If viewed only in the horizontal direction, the horizontal projectile motion will experience uniform motion
\end{SummBox}

Check this \href{https://www.bilibili.com/video/BV1Bz4y167as/}{video} to validate our conclusions.
\begin{figure}[ht]
\centering
\includegraphics[width=0.9\linewidth]{comparison.png}
\caption{Two balls will have exactly same \emph{veritcal displacement}}
\end{figure}

\subsection{Quantitative Equations}
Horizontal Projectile Motion will be better investigated through the view of decomposition and component. Again, acceleration, velocity, and displacement will be studied.

\textbf{Acceleraiton in HP}\\
The horizontal acceleration is \SI{0}{\m\per\square\s}. or we'll say the horizontal motion is uniform motion
The vertical acceleration is \SI{9.81}{\m\per\square\s}, that's why the two balls will have same motion states. Because they are both subject to the \textbf{WEIGHT}

\textbf{Velocity in HP}\\
If the initial velocity is $v$ then, the horizontal component is always:
\begin{equation}
  v_x = v
\end{equation}
While, the vertical can be analyzed as a free fall motion, the vertical component can be calculated as \footnote{assuming downward is positive direction}:
\begin{equation}
  v_y = 0+gt =gt
\end{equation}

If the total velcoity are required, using the vector addtion rule, but don't forget to specify the direction, which is proxied by the angle relative to the horizontal. The formula is:
\begin{align}
  v = \sqrt{v_x^2+v_y^2} &= \sqrt{v^2+(gt)^2}\\
  \theta &= \arctan\left({\frac{v_y}{v_x}}\right)
\end{align}

\textbf{Displacement in HP}\\
Again, calculating the components will be much simpler. The horizontal displacement is:
\begin{equation}
  x = v_x \cdot t  = vt
\end{equation}
The vertical displacement is:
\begin{equation}
  y = \nicefrac{1}{2} \cdot gt^2
\end{equation}
\begin{marginfigure}
\includegraphics{canon.pdf}
\caption{a canon being fired from a cliff}
\label{fig:canon ball}
\end{marginfigure}

If the total displacement\footnote{The path is a curve, thus the distance travelled is a little bit complicated. But with the invention of Calculus, it has become nothing but a mathematical computation problem. If you are interested in this, please contact me} are required, using the vector addition rule, but don't forget to specify the direction, which is proxied by the angle relative to the horizontal. The formula is:
\begin{align}
  s = \sqrt{x^2+y^2} &= \sqrt{(vt)^2+(\nicefrac{1}{2}gt^2)^2}\\
  \alpha &= \arctan\left({\frac{y}{x}}\right)
\end{align}

%visit \href{http://calculuscourse.maa.org/sample/Chapter8/Projects/Length%20of%20a%20curve/length1.html}{Lenght of Curve}
With this, you are ready to solve anything in horizontal projectiles. Let's check one example.
\begin{ExampleBox}
A canon in figure \ref{fig:canon ball} from the top of a cliff, which is \SI{176.58}{\m} high, fires a projectile horizontally with an initial velocity of \SI{75}{\m\per\s}. Find: 
a) The time for which the canon travelled before hitting the ground\\
b) The Horizontal distance travelled.
\tcblower
a) $y=\nicefrac{1}{2} gt^2$, $t=\sqrt{\nicefrac{2h}{g}}=\sqrt{\nicefrac{2\times 176.58}{9.81}}= \SI{6}{\s}$\\
b) $x= vt = 75\times 6 =\SI{450}{\m}$
\end{ExampleBox}

\section{Projectile}
Think about the figure \ref{fig:motorcycle}, the racer and his motor are actually launching with an initial velocity which is \textbf{NOT} horizontal. That seems make things a little bit confusing. But what's important here is to apply the process being utlized to another similar motion process.

\subsection{Decompose the Initial Velocity}
Suppose the racer has an initial velocity of $u$, and it makes an angle of $\theta$ with horizontal. Such initial velocity can be decomposed into two components.
\begin{align}
  u_x &= u\cdot \cos\theta\\
  u_y &= u\cdot \sin\theta
\end{align}

\subsection{Analyze the Process}
In the horizontal direction, the racer will still move in uniform motion with a velocity of $u_x$.

In the vertical direction, the racer will resemble a thrown up motion, which decelerates to maximum height, and then accelerates downward. The initial vertical velocity is $u_y$. the acceleration\footnote{Why using negative sign here? What is the difference with horizontal projectile} is $-g$.

\subsection{Quantitative Equations}
\textbf{Acceleraiton in Projectile}\\
The horizontal acceleration is \SI{0}{\m\per\square\s}. or we'll say the horizontal motion is uniform motion
The vertical acceleration is \SI{-9.81}{\m\per\square\s}

\textbf{Velocity in HP}\\
If the initial velocity is $u$, the angle it makes with the horizontal is $\theta$ then, the horizontal component is \textbf{always}:
\begin{equation}
  v_x = u_x = u\cos\theta
\end{equation}
While, the vertical can be analyzed as a thrown up motion, the vertical component can be calculated as \footnote{assuming upward is postive direction}:
\begin{equation}
  v_y = u_y+(-g)t = u\sin\theta - gt
\end{equation}

If the total velcoity are required, using the vector addtion rule, but don't forget to specify the direction, which is proxied by the angle relative to the horizontal. The formula is:
\begin{align}
  v &= \sqrt{v_x^2+v_y^2} = \sqrt{(u\cos\theta)^2+(u\sin\theta - gt)^2}\\
  \theta &= \arctan\left({\frac{v_y}{v_x}}\right)
\end{align}

\textbf{Displacement in HP}\\
Again, calculating the components will be much simpler. The horizontal displacement is:
\begin{equation}
  x = v_x \cdot t  = (u\cos\theta) \cdot t
\end{equation}
The vertical displacement is:
\begin{equation}
  y = u_yt+\nicefrac{1}{2}\cdot(-g)t^2 = (u\sin\theta)-\nicefrac{1}{2}\cdot gt^2
\end{equation}

The \emph{displacement} is calculated as
\begin{align}
  s &= \sqrt{x^2+y^2}\\
  \alpha &= \arctan{\left(\frac{y}{x}\right)}
\end{align}

\end{document}