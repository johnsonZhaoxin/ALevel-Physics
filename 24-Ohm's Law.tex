\documentclass[a4paper]{tufte-handout}
% for debugging purposes -- displays the margins
%\geometry{showframe}
%\usepackage{float} %不知道和什么冲突了。
\usepackage{amsmath,amsfonts,amssymb,mathtools}
\newcommand{\icol}[1]{% inline column vector
  \left(\begin{smallmatrix}#1\end{smallmatrix}\right)%
}


\newcommand{\irow}[2]{ % 行向量,输出 xi+yj+zk的形式,存在的问题是不能自动根据负数调整为-号
  #1\mathbf{i}+#2\mathbf{j}%+#3\mathbf{k}
}
% \usepackage{geometry} %这个包已经在handout.cls使用过了,如果再调用一次会出问题
% \geometry{left=2cm,right=2cm,top=3cm,bottom=3cm}

%\usepackage[LGR]{fontenc} %使用希腊字符。

\usepackage{siunitx}
\DeclareSIUnit\torr{torr} %声明新的单位torr


\usepackage{BOONDOX-cal} %为了花体的emf符号
\usepackage[normalem]{ulem} %下划线
\usepackage{wrapfig}
\usepackage{caption}
\usepackage{float}
\usepackage[version=4]{mhchem}

%定理的运用
\usepackage[english]{babel}
\newtheorem{theorem}{Theorem}[section]      %定理
\newtheorem{corollary}{Corollary}[theorem]  %结论
\newtheorem{lemma}[theorem]{Lemma}          %引理
\newtheorem{definition}[theorem]{Definition}%定义
% ------------------------------------------------------------------------------
% load hyperref to use hyperlinks
% ------------------------------------------------------------------------------
\usepackage{xcolor}
\definecolor{r1}{HTML}{FF8674}
\definecolor{b1}{HTML}{17ABDD}
\definecolor{p1}{HTML}{D4B6D6}
\definecolor{g1}{HTML}{70E2CB}
\definecolor{o1}{HTML}{DFA743}


\usepackage{hyperref}
\hypersetup{
      colorlinks=true,
      linkcolor=black,
      filecolor=cyan,
      urlcolor=b1,
      citecolor=g1,
}
% Set up the images/graphics package
\usepackage{graphicx}
%\graphicspath{{./auxi/}}
\setkeys{Gin}{width=0.7\linewidth,totalheight=0.3\textheight,keepaspectratio}

% The following package makes prettier tables.  We're all about the bling!
\usepackage{booktabs}

% The units package provides nice, non-stacked fractions and better spacing
% for units.
\usepackage{units}
\usepackage{nicefrac} %比较好看的斜体分号
\usepackage{siunitx}
\sisetup{separate-uncertainty}%产生的效果是是20+3cm 这样
% The fancyvrb package lets us customize the formatting of verbatim
% environments.  We use a slightly smaller font.
\usepackage{fancyvrb}
\fvset{fontsize=\normalsize}

% Small sections of multiple columns
\usepackage{multicol}

% Better variable font
\usepackage{mathptmx}

% todolist
\usepackage{enumitem}
\newlist{todolist}{itemize}{2}
\setlist[todolist]{label=$\square$} %因为美元符号的问题。需要手动添加美元\square美元

\usepackage{tikz}

%beautifulbox
\usepackage{tcolorbox} %带背景色的盒子用于放置Summary,Task,还有Practice
\tcbuselibrary{breakable}
\tcbset{width=\textwidth} %默认盒子的宽度
\newenvironment{TaskBox} %任务盒子
{\begin{tcolorbox}[breakable,colback=b1!30,colframe=b1,title=Task]} {\end{tcolorbox}}
\newenvironment{ExampleBox} %Practice盒子
{\begin{tcolorbox}[breakable,colback=g1!30,colframe=g1,title=Example]} {\end{tcolorbox}}
\newenvironment{SummBox}
{\begin{tcolorbox}[breakable,colback=r1!30,colframe=r1,title=Summary]} {\end{tcolorbox}}


% These commands are used to pretty-print LaTeX commands
%\newcommand{\doccmd}[1]{\texttt{\textbackslash#1}}% command name -- adds backslash automatically
%\newcommand{\docopt}[1]{\ensuremath{\langle}\textrm{\textit{#1}}\ensuremath{\rangle}}% optional command argument
%\newcommand{\docarg}[1]{\textrm{\textit{#1}}}% (required) command argument
%\newenvironment{docspec}{\begin{quote}\noindent}{\end{quote}}% command specification environment
%\newcommand{\docenv}[1]{\textsf{#1}}% environment name
%\newcommand{\docpkg}[1]{\texttt{#1}}% package name
%\newcommand{\doccls}[1]{\texttt{#1}}% document class name
%\newcommand{\docclsopt}[1]{\texttt{#1}}% document class option name

\def\d{{\mathrm{d}}}

% 强制所有段落不缩进
\setlength{\parindent}{0pt}%没有起作用,日
\title{Ohm's Law}
\author{Sanjin Zhao}
\date{2nd Oct, 2022}  % if the \date{} command is left out, the current date will be used


\begin{document}
\maketitle% this prints the handout title, author, and date
%\printclassoptions
\section*{Learning Outcome}
I highly recommend you to finish this checklist to determine whether you've achieved the learning objectives.
\begin{todolist}
  \item state Ohm’s law
  \item sketch and explain the $I$–$V$ characteristics for resistors
  %\item sketch the temperature characteristic for an NTC thermistor
  \item solve problems involving the resistivity of a material.
\end{todolist}
\clearpage

\section{Leadin}
\href{https://www.britannica.com/biography/Georg-Ohm}{Georg Simon Ohm} is a German physicist who is renowned for his great discover between the p.d. and the current in a \emph{resistor}.
\begin{marginfigure}
\includegraphics[width=3.5cm]{auxi/Ohm.jpg}
\caption{Georg Ohm\\1789-1854}
\end{marginfigure}
And in memory of his contribution in this field, the law is called \emph{Ohm's Law} and moreover, his names has been lowercased in order to serve as the unit for the \emph{resistance}.

\section{Ohm's Law}
You might have already been familiar with Ohm's Law, but let's start from scratch in deciding the I-V characteristics of \emph{ohmic resistor}\footnote{def:most metal conductors may be regarded as ohmic resistors}. 
\subsection{Experiment Setup}
The circuit is set up as shown in Fig.\ref{fig:Ohm setup}
\begin{figure}[h]
\centering
\includegraphics[width=0.8\linewidth]{auxi/OhmsLaw.png}
\caption{Setup and circuit diagram of investigation on Ohm's Law}
\label{fig:Ohm setup}
\end{figure}

The voltmeter measures the p.d. across the resistor, while the ammeter measures the current in the whole circuit. By \uline{sliding the \hspace{1in}}, the p.d. across the resistor can be changed, then the currents would change correspondingly. Thus, the $I$-$V$ graph could be drawn. 

\subsection{I-V diagram}
\begin{SummBox}
The $I$-$V$ characteristics is a graph of currents against voltage for a particular component of an electrical circuit
\end{SummBox}
The experiment's results may looks like the following:
\begin{figure}[h]
\centering
\includegraphics[width=0.8\linewidth]{auxi/IVcharacteristic.png}
\caption{The $I$-$V$ characteristics of a resistor}
\end{figure}

\begin{TaskBox}
What does it means when the voltage is negative?
\vspace{0.5in}
\end{TaskBox}

\subsection{Ohm's Law definition}
Since the linearity between currents and voltagee could be found, the Ohm's Law is defined as the following:
\begin{SummBox}
Ohm's Law states that: A conductor(resistor) obeys Ohm's law if the \uline{\hspace{1in}} in it is \uline{directly proportional to} the \uline{\hspace{1in}} across its ends.

If stated in mathematical formulas:
\[
  I\propto V
\]
\end{SummBox}

\subsection{Resistance}
According to Ohm's Law, due to the proportionality between the p.d.s and currents. Resistance of a material can now be defined.
\begin{SummBox}
Electrical Resistance, $R$ , of a resistor is the ratio of potential difference to current.

\[
   R = \nicefrac{V}{I} 
\]
\end{SummBox}
The unit for resistance is \uline{\hspace{1in}}. and thus the SI base unit is \uline{\hspace{1in}}.

\begin{TaskBox}
Combining the definition and the $I$-$V$ characteristics, the gradient of the $I$-$V$ graph is \uline{\hspace{1in}}.
\end{TaskBox}

The relationship between \uline{resistance}, \uline{current}, \uline{voltage or p.d.} can be described humorously in Fig \ref{fig:humor}
\begin{marginfigure}
\centering
\includegraphics[width=4cm]{auxi/humorOhm.png}
\caption{A humorous way to memorize the Ohm's law}
\label{fig:humor}
\end{marginfigure}

\subsection{Power Dissipated on Resistor}
Since the electric power is $P=V\cdot I$ and $V=IR$, thus the power can also be calculated through
\[
   P = I^2\cdot R \qquad P=\nicefrac{V^2}{R}
\]
Any way would be feasible.


\subsection{Resistivity}
The resistance of a metal wire depneds on serveral factors:
\begin{itemize}
  \item length, $L$
  \item cross-sectional area, $A$
  \item the material the wire is made from
  \item the temperature\footnote{this factor will be discussed when superconductivity are mentioned}
\end{itemize}

\subsection{resistivity of a material}
Resistivity is a property of a material, which only changes based on the material itself. The lowest resistivity discovered is the silver, copper ranking the second, the values are (\SI{1.60e-8}{\ohm\m}) and (\SI{1.69e-6}{\ohm\cm})\footnote{Attention, I wrote it on purpose}. This means \uline{(silver/copper)} is the best metal to make conduting wire.

\begin{marginfigure}[2cm]
\includegraphics[width=4cm]{auxi/taobaosilver.png}
\caption{is it an IQ tax when buying silver-coated wire}
\end{marginfigure}


\begin{TaskBox}
Specify qualitatively the affect of each factors, for exmaple: \\
if the length is increased, the wire's resistance will \uline{\hspace{0.5in}}.
\vspace{0.6in}

\tcblower

How rheostat can change the resistance that are connected in a circuit.
\vspace{0.6in}
\end{TaskBox}


\subsection{Determine the resistance}
By controlling variables, the resitance of a wire resistor can be determined by the following formula:
\begin{SummBox}
\[
  R=\frac{\rho L}{A}
\]
\end{SummBox}
Alternatively, the resistivity actually comes from the such definition. 
\[
  \rho=\nicefrac{RA}{L}
\]
thus, the unit for resistivity is usually \uline{\hspace{1in}}. You can also change that unit to SI base unit. 
\end{document}