\documentclass[a4paper]{tufte-handout}
% for debugging purposes -- displays the margins
%\geometry{showframe}
%\usepackage{float} %不知道和什么冲突了。
\usepackage{amsmath,amsfonts,amssymb,mathtools}
\newcommand{\icol}[1]{% inline column vector
  \left(\begin{smallmatrix}#1\end{smallmatrix}\right)%
}
\newcommand{\irow}[2]{ % 行向量,输出 xi+yj+zk的形式,存在的问题是不能自动根据负数调整为-号
  #1\mathbf{i}+#2\mathbf{j}%+#3\mathbf{k}
}
% \usepackage{geometry}
% \geometry{left=2cm,right=2cm,top=3cm,bottom=3cm}

\usepackage{siunitx}
\usepackage[normalem]{ulem}
\usepackage{wrapfig}
\usepackage{caption}
\usepackage{float}



% ------------------------------------------------------------------------------
% load hyperref to use hyperlinks
% ------------------------------------------------------------------------------
\usepackage{xcolor}
\definecolor{r1}{HTML}{FF8674}
\definecolor{b1}{HTML}{17ABDD}
\definecolor{p1}{HTML}{D4B6D6}
\definecolor{g1}{HTML}{70E2CB}
\definecolor{o1}{HTML}{DFA743}

\usepackage{hyperref}
\hypersetup{
      colorlinks=true,
      linkcolor=black,
      filecolor=cyan,
      urlcolor=b1,
      citecolor=green,
}



% Set up the images/graphics package
\usepackage{graphicx}
\setkeys{Gin}{width=0.7\linewidth,totalheight=0.3\textheight,keepaspectratio}
%\graphicspath{{graphics/}}

\title{Error and Uncertainty}
\author{Sanjin Zhao}
\date{1th Sep 2022}  % if the \date{} command is left out, the current date will be used

% The following package makes prettier tables.  We're all about the bling!
\usepackage{booktabs}

% The units package provides nice, non-stacked fractions and better spacing
% for units.
\usepackage{units}

% 国际制单位,使用有点问题好像
\usepackage{siunitx}
\sisetup{separate-uncertainty}%
% The fancyvrb package lets us customize the formatting of verbatim
% environments.  We use a slightly smaller font.
\usepackage{fancyvrb}
\fvset{fontsize=\normalsize}

% Small sections of multiple columns
\usepackage{multicol}

% Better variable font
\usepackage{mathptmx}

% todolist
\usepackage{enumitem}
\newlist{todolist}{itemize}{2}
\setlist[todolist]{label=$\square$}

\usepackage{tikz}

%beautifulbox
\usepackage{tcolorbox} %带背景色的盒子用于放置Summary,Task,还有Practice
\tcbuselibrary{breakable}
\tcbset{width=\textwidth} %默认盒子的宽度
\newenvironment{TaskBox} %任务盒子
{\begin{tcolorbox}[breakable,colback=b1!30,colframe=b1,title=Task]} {\end{tcolorbox}}
\newenvironment{ExampleBox} %Practice盒子
{\begin{tcolorbox}[breakable,colback=g1!30,colframe=g1,title=Example]} {\end{tcolorbox}}
\newenvironment{SummBox}
{\begin{tcolorbox}[breakable,colback=r1!30,colframe=r1,title=Summary]} {\end{tcolorbox}}


% These commands are used to pretty-print LaTeX commands
\newcommand{\doccmd}[1]{\texttt{\textbackslash#1}}% command name -- adds backslash automatically
\newcommand{\docopt}[1]{\ensuremath{\langle}\textrm{\textit{#1}}\ensuremath{\rangle}}% optional command argument
\newcommand{\docarg}[1]{\textrm{\textit{#1}}}% (required) command argument
\newenvironment{docspec}{\begin{quote}\noindent}{\end{quote}}% command specification environment
\newcommand{\docenv}[1]{\textsf{#1}}% environment name
\newcommand{\docpkg}[1]{\texttt{#1}}% package name
\newcommand{\doccls}[1]{\texttt{#1}}% document class name
\newcommand{\docclsopt}[1]{\texttt{#1}}% document class option name


\begin{document}

\maketitle% this prints the handout title, author, and date

%\printclassoptions
\section*{Learning Outcome}
I highly recommend you to finish this checklist to determine whether you've achieved the learning objectives.
\begin{todolist}
	\item Distinguish \textbf{true value} and readings.
	\item Know the difference between \emph{accuracy} and \emph{precision}
	\item Distinguish between \emph{Random Error} and \emph{Systematic Error}\footnote{def:}
	\item State methods to minimize such errors
	\item Express uncertainty in both \emph{absolute} and \emph{relative} forms
	\item Grasp the operation rules for uncertainty
\end{todolist}
\clearpage

\section*{Leadin}
Why so serious?   About sciences.
\begin{marginfigure}
\includegraphics{why so seirous.jpeg}
\caption{Why so serious about science?}
\end{marginfigure}
Despite we want exact value of measurement, but life is so hard. You can not claim your observation or measurement is the \emph{true value}\footnote{def:} of the object being measured. So we need to discuss errors, presions, and uncertainty during measurement\footnote{You can refer to this 
\href{https://www.savemyexams.co.uk/a-level/physics/cie/22/revision-notes/1-physical-quantities--units/1-2-measurements--errors/1-2-1-errors--uncertainties/}{error and uncertainty linkage} to find more information.}.

\section{Accuracy vs Precision}
In physics and chemistry related subjects, Two quality is taken into consideration, Accuracy(Accurate)\footnote{def:The closeness of the measured value to a standard or true value} and Precision(Precise)\footnote{def:The closeness of two or more measurements to each other is known as the precision}.
\begin{center}
\centering
\includegraphics{accurate.pdf}
\caption{Matrix of accuracy and precision}
\label{fig:acc vs pre}
\end{center}

\begin{SummBox}
Try to compare accuracy and precision according to the figure \ref{fig:acc vs pre}
\vspace{0.5in}
\end{SummBox}


\section{Errors}
If we know the exact value, the difference between your reading and the exact value can be explained by \emph{errors}.
\subsection{Random Error}
\begin{itemize}
	\item Random errors cause \textbf{unpredictable} fluctuations in an instrument’s readings as a result of uncontrollable factors, such as environmental conditions
	\item This affects the \textbf{precision of the measurements} taken, causing a wider spread of results about the mean value
\end{itemize}
In figure \ref{fig:acc vs pre}, all of them are showing random error.


\subsection{Systematic Error}
\begin{itemize}
	\item Systematic errors arise from the use of faulty instruments used or from flaws in the experimental method.
	\item Systematic errors are biased\footnote{which means it always increased or decreased the reading value}
	\item This type of error is repeated every time the instrument is used or the method is followed, which affects the \textbf{accuracy of all readings} obtained
\end{itemize}
In figure \ref{fig:acc vs pre}, (a) and (c) are showing systematic error.

\subsection{Zero Error}
Zero error is a classic systematic error, in which an instrument gives a reading when the true reading at that time is zero. This can be mitigated by \textbf{recalibrating} the instrument\footnote{ususally press the zero button}.

\begin{SummBox}
List the methods that random error and systematic error can be reduced or eliminated
\vspace{1in}
\end{SummBox}

\section{Uncertainty}
In order to solve the problem of reading, uncertainty is introduced. Just in the marginfigure, percentage or absolute uncertainty is introduced in labelling the mass or the volume of foods
\begin{marginfigure}
\includegraphics[width=5cm]{shortage.jpeg}
\caption{shortage are allowed in foods labelling}
\end{marginfigure}
It is usually be expressed by:
\begin{center}
	reading value $\pm$ uncertainty
\end{center}
For example, if you measure the mass of an apple, with uncertainty of \SI{1}{\gram}. The result would be:
\begin{center}
		\SI[multi-part-units = single]{200(1)}{\gram}
\end{center}
This means the \textbf{true value} of the mass of the apple is some value within the range of \SIrange{199}{201}{\gram}.

\subsection{Absolute uncertainty}
The above is an expression for \textbf{absolute uncertainty}. And usually the absolutely uncertainty is defined as half of the precision of the instrument. For example the precision of the micrometer is \SI{0.01}{\milli\metre}, thus the uncertainty is  \SI{0.005}{\milli\metre}

\subsection{Fractional uncertainty}
Another form is \textbf{fractional uncertainty}, which gives the perception of the uncertainty relative to the reading. For example, The reading of the apple could also be written as:
\begin{center}
	\SI{200}{\gram} $\pm 0.5\%$
\end{center}

\begin{SummBox}
How the absolute uncertainty \SI{1}{\gram} can be changed into percentage uncertainty
\vspace{0.5in}
\end{SummBox}

\section{Operation Rules}
The last thing is to calculate the combined uncertainty when dealing with multiple readings. The following are some rules to be utilized.
\subsection{Addition or Subtraction}
Serveral thing have to be kept in mind to deal with the addition or subtraction of uncertainty.
\begin{enumerate}
	\item absolute uncertainty should be used, that means if percentage uncertainty is used, you must convert it before calculation, the fomula for converting is:
	\[
		\text{percentage uncertainty} = \frac{\hspace{1.5  in}}{} \times 100\%
	\]
	\item no matter addition or subtraction, always \textbf{ADD} the uncertainties to determine the absolute uncertainty of the final result
\end{enumerate}

\begin{ExampleBox}
The length and width of an rectangle, with uncertainty, are measured to be \SI[multi-part-units = single]{4.0(4)}{\m} and \SI[multi-part-units = single]{3.0(3)}{\m}. What is the perimeter with uncertainty
\vspace{0.5in}
\end{ExampleBox}

\subsection{Multiplication or Division}
To determine the uncertainty of products or quotients, \textbf{Percentage Uncertainty} should be used
\begin{enumerate}
	\item No matter multiplication or division, always \textbf{ADD} the uncertainties to determine the absolute uncertainty of the final result.
	\item if not specified, the final result can be expressed in percentage. However, if absolute uncertainty is required, the formula for converting is:
	\[
		\text{absolute uncertainty} = \frac{\hspace{1.5  in}}{}
	\]
\end{enumerate}

If any physical quantites is multiplied or divided by an exact value or scientific constant. the final result would share same percentage uncertainty as orignial. 

\begin{ExampleBox}
The length and width of an rectangle, with uncertainty, are measured to be \SI[multi-part-units = single]{4.0(4)}{\m} and \SI[multi-part-units = single]{3.0(3)}{\m}. What is the area with uncertainty
\vspace{0.5in}
\tcblower
The radius of a circle is measured to be \SI[multi-part-units = single]{10.0(5)}{\m}. What is the circumference of the cirle with uncertainty?
\vspace{0.5in}
\end{ExampleBox}


\subsection{Powers or Roots}
Since powers can be viewd as advanced multiplication, it follows the rule of multiplication. For example, the reading is $x$ and the absolute uncertainty is $\Delta x$, if $r = x^n$, then
\[
	\frac{\Delta r}{r} = n \times \frac{\Delta x}{x} \footnote{That is because $r=\underbrace{x\cdot x \cdot \ldots\cdot x}_{n}$, the percentage of each term should be added}
\]

\begin{ExampleBox}
The radius of a circle is measured to be \SI[multi-part-units = single]{10.0(5)}{\m}. What is the area of the cirle with uncertainty?
\vspace{0.5in}
\end{ExampleBox}

And because of the law of exponent\footnote{Ask your mathematics teacher}. $\sqrt[n]{x} = x^\frac{1}{n}$. If $r=\sqrt[n]{x}$, then:
\[
	\frac{\Delta}{r} = \frac{1}{n} \cdot \frac{\Delta x}{x}
\]

\begin{ExampleBox}
The area of a square is measured to be \SI{25.0}{\square\m} with 10\% of uncertainty. What is the side length of the square and its corresponding uncertainty?
\vspace{0.5in}
\tcblower
The side length of a square is measured to be \SI{5.0}{\m} with 10\% of uncertainty. What is the area of the square and its corresponding uncertainty?
\vspace{0.5in}
\end{ExampleBox}
\end{document}