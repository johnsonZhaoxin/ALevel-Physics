\documentclass[12pt,a4paper]{tufte-handout}
% for debugging purposes -- displays the margins
%\geometry{showframe}
%\usepackage{float} %不知道和什么冲突了。
\usepackage{amsmath,amsfonts,amssymb,mathtools}
\usepackage{siunitx}
\usepackage[normalem]{ulem}



% ------------------------------------------------------------------------------
% load hyperref to use hyperlinks
% ------------------------------------------------------------------------------
\usepackage{xcolor}
\definecolor{r1}{HTML}{FF8674}
\definecolor{b1}{HTML}{17ABDD}
\definecolor{p1}{HTML}{D4B6D6}
\definecolor{g1}{HTML}{70E2CB}
\definecolor{o1}{HTML}{DFA743}

\usepackage{hyperref}
\hypersetup{
      colorlinks=true,
      linkcolor=black,
      filecolor=cyan,
      urlcolor=b1,
      citecolor=green,
}



% Set up the images/graphics package
\usepackage{graphicx}
\setkeys{Gin}{width=\linewidth,totalheight=\textheight,keepaspectratio}
%\graphicspath{{graphics/}}

\title{Preparations for A-Level Physics}
\author{Sanjin Zhao}
\date{1th Sep 2022}  % if the \date{} command is left out, the current date will be used

% The following package makes prettier tables.  We're all about the bling!
\usepackage{booktabs}

% The units package provides nice, non-stacked fractions and better spacing
% for units.
\usepackage{units}

% 国际制单位,使用有点问题好像
\usepackage{siunitx}
% The fancyvrb package lets us customize the formatting of verbatim
% environments.  We use a slightly smaller font.
\usepackage{fancyvrb}
\fvset{fontsize=\normalsize}

% Small sections of multiple columns
\usepackage{multicol}

% Better variable font
\usepackage{mathptmx}

% todolist
\usepackage{enumitem}
\newlist{todolist}{itemize}{2}
\setlist[todolist]{label=$\square$}

\usepackage{tikz}

%beautifulbox
\usepackage{tcolorbox} %带背景色的盒子用于放置Summary,Task,还有Practice
\tcbuselibrary{breakable}
\tcbset{width=\textwidth} %默认盒子的宽度
\newenvironment{TaskBox} %任务盒子
{\begin{tcolorbox}[breakable,colback=b1!30,colframe=b1,title=Task]} {\end{tcolorbox}}
\newenvironment{ExampleBox} %Practice盒子
{\begin{tcolorbox}[breakable,colback=g1!30,colframe=g1,title=Example]} {\end{tcolorbox}}
\newenvironment{SummBox}
{\begin{tcolorbox}[breakable,colback=r1!30,colframe=r1,title=Summary]} {\end{tcolorbox}}


% These commands are used to pretty-print LaTeX commands
\newcommand{\doccmd}[1]{\texttt{\textbackslash#1}}% command name -- adds backslash automatically
\newcommand{\docopt}[1]{\ensuremath{\langle}\textrm{\textit{#1}}\ensuremath{\rangle}}% optional command argument
\newcommand{\docarg}[1]{\textrm{\textit{#1}}}% (required) command argument
\newenvironment{docspec}{\begin{quote}\noindent}{\end{quote}}% command specification environment
\newcommand{\docenv}[1]{\textsf{#1}}% environment name
\newcommand{\docpkg}[1]{\texttt{#1}}% package name
\newcommand{\doccls}[1]{\texttt{#1}}% document class name
\newcommand{\docclsopt}[1]{\texttt{#1}}% document class option name


\begin{document}

\maketitle% this prints the handout title, author, and date

%\printclassoptions
\section*{Learning Outcome}
I highly recommend you to finish this checklist to determine whether you've achieved the leanring objectives.

\begin{todolist}
	\item Understand that all physical quantities consist of a numerical magnitude and a unit.
	\item Recall the following SI \textbf{base quantities} and their units:
	\item Recall and use \textbf{prefixes} and their symbols
	\item Make reasonable estimates of physical quantities included within the syllabus
	\item Express \textbf{derived units} as products or quotients of the SI base units
	\item Use SI base units to check the \emph{homogeneity}\footnote{def:}
	% \item Understand the difference between scalar and vector quantities
	% \item Add and subtract coplanar vectors
	% \item Represent a vector as two perpendicular components
\end{todolist}
\clearpage


\section{Leadin}
Watch the \href{https://www.youtube.com/watch?v=Xpn9eTNZiCs}{videos} and try to answer the following questions:
\begin{marginfigure}
	\includegraphics{SI logo.png}
	\caption{SI logo}
\end{marginfigure}
\begin{TaskBox}
\begin{enumerate}
	\item what is the meaning of unifying the measuring system?
	\item any revolution happens when defining the base units? If so, what is the change?
\end{enumerate}
\end{TaskBox}

\section{SI system}
\subsection*{Base Unit}
\href{https://www.bipm.org/en/measurement-units}{BIPM} has established the decimal as the common system and defined the orginal three \emph{base units} in 1875. And with the development of science and technology, 4 more base units are included into the system and finally become the \emph{Le Système international d’unités}\footnote{from French}. What are they?
\begin{TaskBox}
Finish the 7 base units in the following, stating what they measures and the units.
\begin{table}
\centering
\begin{tabular}{|l|c|l|c|}
\hline
\multicolumn{2}{|l|}{Base Quantity}   & \multicolumn{2}{l|}{Base Unit} \\ \hline
name                      & symbol    & name             & unit        \\ \hline
                      & $t$         &            &            \\ \hline
                    &  & meter            &           \\ \hline
Mass                      & $m$         &         &           \\ \hline
         & $I$,$i$      &            &           \\ \hline
 & $T$         &           &            \\ \hline
       & $n$         &             &          \\ \hline
Luminous intensity        & $I_v$      &           &           \\ \hline
\end{tabular}
\end{table}
\tcblower
Extended Questions:\\
Why using constants rather than realistic items to define 1 base unit?
\end{TaskBox}

\subsection*{Prefix of SI}
However, one base unit may not always suitable to measure tiny or larger objects. For example, 1 metre may not be appropriate to measure the distance a train travels, and neither suitable for the length of a bacterium.
\begin{marginfigure}
	\includegraphics{scanning electroscope.jpeg}
	\caption{Electroscope of bacteria}
\end{marginfigure}
there are several prefix either enlarge or minimize the base unit, most common among which are kilo and centi.
You have to memorize the conversion factor and symbol of them.
\begin{table}[H]
\centering
\begin{tabular}{|l|l|l|l|l|l|}
\hline
Factor    & Name  & Symbol & Factor    & Name  & Symbol \\ \hline
$10^{\square}$ &   & T      & $10^{\square}$ & deci  &       \\ \hline
$10^{\square}$    &   & G      & $10^{\square}$ & centi &       \\ \hline
$10^{\square}$    &   & M      & $10^{\square}$ & milli & m      \\ \hline
$10^{\square}$    &   & k      & $10^{\square}$ &  & $\mu$   \\ \hline
$10^{\square}$    &  & h      & $10^{\square}$ & nano  & n      \\ \hline
$10^{\square}$ &   & da     & $10^{\square}$ &   & p      \\ \hline
\end{tabular}
\end{table}


\subsection*{Derived Units}
Apparently, 7 base unit would not be enough to measuring every quantities in the physics world. Other units are also important, for example, newton\footnote{not CAPITALIZE, to show that it is a unit not a person} is the most common units when measuring forces. So why only the 7 units are called `\emph{BASE UNIT}', it is because that all other units can be \textbf{derived} from them. Such units are called derived units.

The principle that the derived units can de deduced is that:\\
\uline{\hspace{5 in}}

Find formula of common derived units:
\begin{table}
\begin{tabular}{|l|c|c|c|c|}
\hline
Quantity  & \multicolumn{1}{l|}{name} & \multicolumn{1}{l|}{unit} & \multicolumn{1}{l|}{dimension} & \multicolumn{1}{l|}{formula} \\ \hline
Force     & newton                    & N                         &$\si{kg.m.s^{-2}}$                                &   $F=ma$                           \\ \hline
Pressure  &  \hspace{0.7 in}      &                           &                                &   \hspace{0.7 in}                \\ \hline
Energy    &                           &                           &                                &                              \\ \hline
Frequency &                           &                           &                                &                              \\ \hline
Power     &                           &                           &                                &                              \\ \hline
Momentum  &                           &                           &                                &                              \\ \hline
Impulse   &                           &                           &                                &                              \\ \hline
Torque    &                           &                           &                                &                              \\ \hline
\end{tabular}
\end{table}


\subsection*{Conversion and Dimension Analysis}
What if I want to change the distance that a Tesla Model 3 can travel from \emph{Metric System}\footnote{def:} into the \emph{Imperial System}?
\begin{marginfigure}
	\includegraphics{Tesla Model3.png}
	\caption{Lying Travel Distance}
\end{marginfigure}
A common way to convert is using fraction multiplication.


\subsection*{Approximate values of common objects}
Try to finish the following tables to memorize the common estimated phyical quantities.
\begin{table}[t]
\begin{tabular}{|l|l|}
\hline
Mass of a person                             & \uline{\hspace{1 in}} \\ \hline
Height of a person                           & \uline{\hspace{1 in}} \\ \hline
Walking speed                          	   & \uline{\hspace{1 in}} \\ \hline
Speed of a car on the motorway 		   & \uline{\hspace{1 in}} \\ \hline
Volume of a can of a drink                   & \uline{\hspace{1 in}} \\ \hline
Density of water                             & \uline{\hspace{1 in}} \\ \hline
Density of air                               & \uline{\hspace{1 in}} \\ \hline
Mass of an apple                             & \uline{\hspace{1 in}} \\ \hline
e.m.f of a car battery                       & \uline{\hspace{1 in}} \\ \hline
e.m.f of a AA battery                        & \uline{\hspace{1 in}} \\ \hline
Current in a domestic appliance              & \uline{\hspace{1 in}} \\ \hline
Hearing range                                & \uline{\hspace{1 in}} \\ \hline
Young’s Modulus of a material                & \uline{\hspace{1 in}} \\ \hline
\end{tabular}
\end{table}


\end{document}