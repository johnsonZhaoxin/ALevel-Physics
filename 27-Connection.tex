\documentclass[a4paper]{tufte-handout}
% for debugging purposes -- displays the margins
%\geometry{showframe}
%\usepackage{float} %不知道和什么冲突了。
\usepackage{amsmath,amsfonts,amssymb,mathtools}
\newcommand{\icol}[1]{% inline column vector
  \left(\begin{smallmatrix}#1\end{smallmatrix}\right)%
}


\newcommand{\irow}[2]{ % 行向量,输出 xi+yj+zk的形式,存在的问题是不能自动根据负数调整为-号
  #1\mathbf{i}+#2\mathbf{j}%+#3\mathbf{k}
}
% \usepackage{geometry} %这个包已经在handout.cls使用过了,如果再调用一次会出问题
% \geometry{left=2cm,right=2cm,top=3cm,bottom=3cm}

%\usepackage[LGR]{fontenc} %使用希腊字符。

\usepackage{siunitx}
\DeclareSIUnit\torr{torr} %声明新的单位torr


\usepackage{BOONDOX-cal} %为了花体的emf符号
\usepackage[normalem]{ulem} %下划线
\usepackage{wrapfig}
\usepackage{caption}
\usepackage{float}
\usepackage[version=4]{mhchem}

%定理的运用
\usepackage[english]{babel}
\newtheorem{theorem}{Theorem}[section]      %定理
\newtheorem{corollary}{Corollary}[theorem]  %结论
\newtheorem{lemma}[theorem]{Lemma}          %引理
\newtheorem{definition}[theorem]{Definition}%定义
% ------------------------------------------------------------------------------
% load hyperref to use hyperlinks
% ------------------------------------------------------------------------------
\usepackage{xcolor}
\definecolor{r1}{HTML}{FF8674}
\definecolor{b1}{HTML}{17ABDD}
\definecolor{p1}{HTML}{D4B6D6}
\definecolor{g1}{HTML}{70E2CB}
\definecolor{o1}{HTML}{DFA743}


\usepackage{hyperref}
\hypersetup{
      colorlinks=true,
      linkcolor=black,
      filecolor=cyan,
      urlcolor=b1,
      citecolor=g1,
}
% Set up the images/graphics package
\usepackage{graphicx}
%\graphicspath{{./auxi/}}
\setkeys{Gin}{width=0.7\linewidth,totalheight=0.3\textheight,keepaspectratio}

% The following package makes prettier tables.  We're all about the bling!
\usepackage{booktabs}

% The units package provides nice, non-stacked fractions and better spacing
% for units.
\usepackage{units}
\usepackage{nicefrac} %比较好看的斜体分号
\usepackage{siunitx}
\sisetup{separate-uncertainty}%产生的效果是是20+3cm 这样
% The fancyvrb package lets us customize the formatting of verbatim
% environments.  We use a slightly smaller font.
\usepackage{fancyvrb}
\fvset{fontsize=\normalsize}

% Small sections of multiple columns
\usepackage{multicol}

% Better variable font
\usepackage{mathptmx}

% todolist
\usepackage{enumitem}
\newlist{todolist}{itemize}{2}
\setlist[todolist]{label=$\square$} %因为美元符号的问题。需要手动添加美元\square美元

\usepackage{tikz}
\usepackage{circuitikz}

%beautifulbox
\usepackage{tcolorbox} %带背景色的盒子用于放置Summary,Task,还有Practice
\tcbuselibrary{breakable}
\tcbset{width=\textwidth} %默认盒子的宽度
\newenvironment{TaskBox} %任务盒子
{\begin{tcolorbox}[breakable,colback=b1!30,colframe=b1,title=Task]} {\end{tcolorbox}}
\newenvironment{ExampleBox} %Practice盒子
{\begin{tcolorbox}[breakable,colback=g1!30,colframe=g1,title=Example]} {\end{tcolorbox}}
\newenvironment{SummBox}
{\begin{tcolorbox}[breakable,colback=r1!30,colframe=r1,title=Summary]} {\end{tcolorbox}}


% These commands are used to pretty-print LaTeX commands
%\newcommand{\doccmd}[1]{\texttt{\textbackslash#1}}% command name -- adds backslash automatically
%\newcommand{\docopt}[1]{\ensuremath{\langle}\textrm{\textit{#1}}\ensuremath{\rangle}}% optional command argument
%\newcommand{\docarg}[1]{\textrm{\textit{#1}}}% (required) command argument
%\newenvironment{docspec}{\begin{quote}\noindent}{\end{quote}}% command specification environment
%\newcommand{\docenv}[1]{\textsf{#1}}% environment name
%\newcommand{\docpkg}[1]{\texttt{#1}}% package name
%\newcommand{\doccls}[1]{\texttt{#1}}% document class name
%\newcommand{\docclsopt}[1]{\texttt{#1}}% document class option name

\def\d{{\mathrm{d}}}

% 强制所有段落不缩进
\setlength{\parindent}{0pt}%没有起作用,日
\title{Series and Parallel}
\author{Sanjin Zhao}
\date{2nd Oct, 2022}  % if the \date{} command is left out, the current date will be used


\begin{document}
\maketitle% this prints the handout title, author, and date
%\printclassoptions
\section*{Learning Outcome}
I highly recommend you to finish this checklist to determine whether you've achieved the learning objectives.
\begin{todolist}
  \item use Kirchhoff’s laws to derive the formulae for the combined resistance of two or more resistors \emph{in series and in parallel}
  \item recognise that ammeters are connected in series within a circuit and therefore should have low resistance
  \item recognise that voltmeters are connected in parallel across a component, or components, and therefore should have high resistance.
\end{todolist}
\clearpage

\section{Leadin}
In mechanics, if two springs with spring constant $k_1$ and $k_2$ are connected in parallel, the equivalent spring constant $k_{eq}=k_1+k_2$. If connected in series, the equivalent spring consant can be determined by this formula:\uline{\hspace{1in}}.
\begin{marginfigure}
\includegraphics[width=5cm]{auxi/connectionofspring.jpg}
\caption{Specify the connection types}
\end{marginfigure}
Now we are going to discuss the connection of two resistors.

\section{Series Connection}
Two resistors are connected in series if one resistor is connected one after another, as shown in the Fig.\ref{fig:series}. 
\begin{marginfigure}
\includegraphics[width=5cm]{auxi/seriesconnection.jpg}
\caption{Two resistors in series}
\label{fig:series}
\end{marginfigure}
The current flows out of $R_1$ and then flows into $R_2$, since there is just one junction between $R_1$ and $R_2$, based on KCL, the conclusion in series circuits are:
\begin{center}
currents are equal in all resistors in series
\end{center}

If one equivalent resistor are used to replace the combination of the two resistors, the requirements should be the following:
\begin{itemize}
  \item p.d. across the equivalent resistor should be equal to the \textsc{total} p.d.s of the two
  \item the current flowing in the equivalent resistor should be equal to that of two resistors.
\end{itemize}

\begin{ExampleBox}
According to the requirement and Fig.\ref{fig:series}, deduce the $R_{eq}$ in terms of $R_1$ and $R_2$
\tcblower
$V=I\cdot R_1+I\cdot R_2$, and in the equivalent resistor, $V=I\cdot R_{eq}$, thus
\[
  R_{eq} =\frac{V}{I}= \frac{I\cdot R_1+I\cdot R_2}{I} = \uline{\hspace{1in}}
\]
\end{ExampleBox} 

And this rule can be applied further, if more than two resistors are connected in the circuits\footnote{Three laws are applied in determining this, state which laws they are.}, such process can be repeated. 

\section{Parallel Connection}
Two resistors are connected in parallel if the wire seems parallel, or they are connected to the same junctions in the circuits, as shown in the Fig.\ref{fig:parallel}. 
\begin{marginfigure}
\includegraphics[width=5cm]{auxi/parallelconnection.jpg}
\caption{Two resistors in parallel}
\label{fig:parallel}
\end{marginfigure}

The currents will seprate into two individual ways, one flows into $R_1$,denoting it as $I_1$, and one flows into $R_2$, denoting it as $I_2$. since the two resistors are using common junctions, the potential difference no matter viewed from either resistor.   
\begin{center}
p.d.s are equal in all resistors in parallel
\end{center}

Again, if one equivalent resistor are used to replace the combination of the two resistors, the requirements should be the following:
\begin{itemize}
  \item p.d. across the equivalent resistor should be equal to the \textsc{total} p.d.s of the two
  \item the current flowing in the equivalent resistor should be equal to that of two resistors.
\end{itemize}

\begin{ExampleBox}
According to the requirement and Fig.\ref{fig:parallel}, deduce the $R_{eq}$ in terms of $R_1$ and $R_2$
\tcblower
$V=I_1\cdot R_1=I_2\cdot R_2$, and in the equivalent resistor, $V=I\cdot R_{eq}$, thus
\[
  R_{eq} =\frac{V}{I}= \frac{V}{I_1+I_2} = \uline{\hspace{1in}}
\]
\end{ExampleBox} 


\begin{SummBox}
Total(equivalent) resistance in series connection is given by the equation:
\[
  R_{eq} = R_1+ R_2 +R_3+\ldots
\]
Total(equivalent) resistance in parallel connection is given by the equation:
\[
  \frac{1}{R_{eq}} = \frac{1}{R_{1}}+\frac{1}{R_{2}}+\frac{1}{R_{3}}+\ldots
\]
\end{SummBox}

\begin{TaskBox}
Compare that to the connection types of spring, think why similarities exsit. 
\end{TaskBox}

\section{Summary}
In a more complex circuits, resistors are connected in various types, the most important thing is the find the \uline{smallest unit of connection}.
\begin{figure}[h]
\centering
\includegraphics[width=0.5\linewidth]{auxi/bonuscircuit.png}
\caption{a more complicated circuits}
\end{figure}

\begin{ExampleBox}
Without using Kirchhoff's Law, Find the total resistance between P and Q.
\tcblower
The smallest unit of connection is \uline{\hspace{1in}}, and the equivalent resistance is \SI{3}{\ohm}, then the circuit diagram can be simplified into the following
\vspace{1in}
Then, repeat the procedure. 
\end{ExampleBox}


\end{document}